\input{./econtexRoot}\input{\econtexRoot/econtexPaths}\documentclass[titlepage]{\econtex}\providecommand{\texname}{BufferStockTheory}


\providecommand{\EqDir}{Equations}
\providecommand{\FigDir}{Figures}
\providecommand{\CodeDir}{Code}
\providecommand{\CalibrationDir}{Calibration}
\providecommand{\TableDir}{Tables}
\providecommand{\ApndxDir}{Appendices}

\usepackage{subfiles}

\providecommand{\onlyinsubfile}{}
\providecommand{\notinsubfile}{}
\renewcommand{\onlyinsubfile}[1]{}
\renewcommand{\notinsubfile}[1]{#1} 


\usepackage{\econtexSetup}\usepackage{\econtexShortcuts}\usepackage{makecell} 
\input{./econtexRoot}\input{\econtexRoot/econtexPaths}\documentclass[titlepage]{\econtex}\providecommand{\texname}{BufferStockTheory}


\providecommand{\EqDir}{Equations}
\providecommand{\FigDir}{Figures}
\providecommand{\CodeDir}{Code}
\providecommand{\CalibrationDir}{Calibration}
\providecommand{\TableDir}{Tables}
\providecommand{\ApndxDir}{Appendices}

\usepackage{subfiles}

\providecommand{\onlyinsubfile}{}
\providecommand{\notinsubfile}{}
\renewcommand{\onlyinsubfile}[1]{}
\renewcommand{\notinsubfile}[1]{#1} 


\usepackage{\econtexSetup}\usepackage{\econtexShortcuts}\usepackage{makecell} 
\input{./econtexRoot}\input{\econtexRoot/econtexPaths}\documentclass[titlepage]{\econtex}\providecommand{\texname}{BufferStockTheory}


\providecommand{\EqDir}{Equations}
\providecommand{\FigDir}{Figures}
\providecommand{\CodeDir}{Code}
\providecommand{\CalibrationDir}{Calibration}
\providecommand{\TableDir}{Tables}
\providecommand{\ApndxDir}{Appendices}

\usepackage{subfiles}

\providecommand{\onlyinsubfile}{}
\providecommand{\notinsubfile}{}
\renewcommand{\onlyinsubfile}[1]{}
\renewcommand{\notinsubfile}[1]{#1} 


\usepackage{\econtexSetup}\usepackage{\econtexShortcuts}\usepackage{makecell} 
\input{./econtexRoot}\input{\econtexRoot/econtexPaths}\documentclass[titlepage]{\econtex}\providecommand{\texname}{BufferStockTheory}


\providecommand{\EqDir}{Equations}
\providecommand{\FigDir}{Figures}
\providecommand{\CodeDir}{Code}
\providecommand{\CalibrationDir}{Calibration}
\providecommand{\TableDir}{Tables}
\providecommand{\ApndxDir}{Appendices}

\usepackage{subfiles}

\providecommand{\onlyinsubfile}{}
\providecommand{\notinsubfile}{}
\renewcommand{\onlyinsubfile}[1]{}
\renewcommand{\notinsubfile}[1]{#1} 


\usepackage{\econtexSetup}\usepackage{\econtexShortcuts}\usepackage{makecell} 
\input{\econtexRoot/BufferStockTheory.sty}

\provideboolean{Shorter}
\setboolean{Shorter}{true}
\setboolean{Shorter}{false}
\providecommand{\ShorterYN}{\ifthenelse{\boolean{Shorter}}}
\usepackage{rotating}\usepackage{subfigure}


\hypersetup{pdfauthor={William Du <wdu9@jhu.edu>},
            pdftitle={Theoretical Foundations of Buffer Stock Saving},
            pdfkeywords={Precautionary saving, buffer-stock saving, consumption, marginal propensity to consume, permanent income hypothesis},
            pdfcreator = {wdu9@jhu.edu}
}

\begin{document}\bibliographystyle{\econtexBibStyle}
\renewcommand{\onlyinsubfile}[1]{}\renewcommand{\notinsubfile}[1]{#1} 

\hfill{\tiny \texname.tex, \today}

\begin{verbatimwrite}{\texname.title}
Theoretical Foundations of Buffer Stock Saving
\end{verbatimwrite}


\title{Distribution of Wealth and Monetary Policy}

\author{William Du\authNum}

\keywords{Precautionary saving, Heterogeneous Agents, Monetary Policy, permanent income hypothesis}

\jelclass{D81, D91, E21}


\maketitle 


\hypertarget{abstract}{}
\begin{abstract}
  This paper develops a heterogenous Agent New Keynesian Model with a friedman buffer stock income process.
\end{abstract}

\begin{small}
\parbox{\textwidth}{
\begin{center}
\begin{tabbing}
\texttt{~Archive:~} \= \= \url{} \kill \\  %
\texttt{~~~~~PDF:~} \> \> \url{} \\
\texttt{~~Slides:~} \> \> \url{} \\
\texttt{~~~~~Web:~} \> \> \url{}    \\
\texttt{~~GitHub:~} \> \> \url{http://github.com/wdu9/FBS-NK} \\
\texttt{~~~~~~~~~~} \> \> \textit{(In GitHub repo, see \texttt{/Code} for tools for solving and simulating the model)} \\
\end{tabbing}
\end{center}
          
\href{https://colab.research.google.com/github/econ-ark/REMARK/blob/master/REMARKs/BufferStockTheory/BufferStockTheory.ipynb}{CLICK HERE} for an interactive \href{http:https://en.wikipedia.org/wiki/Project_Jupyter}{Jupyter Notebook} that uses the \href{https://econ-ark/HARK}{Econ-ARK/HARK} toolkit (\cite{carroll_et_al-proc-scipy-2018}) to produce all of the paper's figures (warning: it may take several minutes to launch)
}
\end{small}

\begin{authorsinfo}
\name{Contact: \href{mailto:wdu9@jhu.edu}{\texttt{wdu9@jhu.edu}}}
\end{authorsinfo}

\thanks{Thanks to }

\titlepagefinish


\newtheorem{defn}{Definition}
\newtheorem{theorem}{Theorem}

\hypertarget{Introduction}{}
\section{Introduction}

\label{sec:intro}


Write here for intro



\hypertarget{The-Model}{}
\section{The Model}

\subsection{Households}
\label{subsec:Households} 

There is a continuum of households of mass 1 distributed on the unit
interval and indexed by $i$. Households are ex-ante heterogeneous in their discount factors and are subject to idiosyncratic income shocks.  Each household faces the following problem:

\begin{verbatimwrite}{\EqDir/supfn.tex}
\begin{eqnarray}
  \label{eq:supfn}
  \max_{\{\cLevBF_{it+s}\}_{s=0}^{\infty}} \mathrm{E_{t}}\left[\sum_{s=0}^{\infty} (\not D \beta_{i})^{t+s} U\left(  \cLevBF_{i t+s}, n_{i t+s}\right)\right]
\end{eqnarray}
\end{verbatimwrite}
\input{\EqDir/supfn.tex} 

subject to 
\begin{align*}
\aLevBF_{it}     &= \mLevBF_{it} - \cLevBF_{it}   \label{eq:DBCparts} \\
\aLevBF_{it} +\cLevBF_{it}    &= \mathbf{z}_{it} +   (1 + r^{a}_{t} ) \aLevBF_{it-1}   \\ 
\aLevBF_{it}  &\geq 0 \\
\end{align*}

where
$U\left(\cLevBF_{i t}, n_{i t}\right) = \frac{\cLevBF_{i t}^{1-\rho}}{1 -\rho} - \varphi \pLevBF_{it} \frac{n_{it}^{1+v}}{1+v}$ , $\beta_{i}$ is the discount factor of household $i$ and $\not D$ is the probability of death.  \\

$\mLevBF_{it}$ \ denotes household $i$'s market resources at time $t$ to be expended on consumption $\cLevBF_{it}$ or invested into an asset $ \aLevBF_{it}$ with return $r_{t+1}^{a}$.  $\mLevBF_{it}$ is determined by labor income,  $\mathbf{z}_{it}$, and the gross return on assets from the last period, $(1+r_{t}^{a}) \aLevBF_{it-1} $. Labor supply of household $i$ at time $t$ is denoted by $n_{it}$.  Given the formulation of sticky wages described in section 2.4, labor supply is an aggregate state variable and therefore consumption serves as the sole control variable in the dynamic problem. \\




\begin{align*}
\mathbf{z}_{it} &= \pLevBF_{it}\tShkAll_{it} \\
\pLevBF_{it+1} &=\pLevBF_{it} \pShk_{it+1} \\
\end{align*}


Labor income is subject to permanent and transitory idiosyncratic shocks. In particular, household $i$'s labor income is composed of a permanent component, $\pLevBF_{it} $ indicating the level of permanent income and a transitory component, $\tShkAll_{it} $, indicating the transitory income shock received by household $i$ at time $t$. $\pLevBF_{it} $ is subject to permanent income shocks $\pShk_{it+1}$ where $\pShk_{it}$ is iid mean one lognormal with standard deviation $\sigma_\pShk$, $\forall t$  
($\Ex_{t}[{\pShk}_{t+n}]=1~\forall~n>0$) .



The transitory random variable follows   
\begin{verbatimwrite}{\EqDir/tShkDef}
\begin{equation}
\tShkAll _{it+n}=
\begin{cases}
 u \phantom{_{t+1}/\pNotZero} & \text{with probability $\pZero>0$} \\
 \tShkEmp_{it+n} (1-\tau_{t})\int_{0}^{1} w_{gt}n_{igt} \, dg      & \text{with probability $\pNotZero  $} 
\end{cases} \label{eq:tShkDef}
\end{equation}
\end{verbatimwrite}
\input{\EqDir/tShkDef.tex}
where $\tau_{t}$ is the tax rate , $w_{gt}$ is the real wage for labor type $g$ at time t, $ n_{igt}$ is the labor supply for labor type $g$ and $\tShkEmp_{t+n}$ is an iid mean-one lognormal with standard deviation $\sigma_{\tShkEmp}$,
($\Ex_{t}[{\tShkEmp}_{t+n}]=1~\forall~n>0$).




\begin{comment}
Combining the transition equations, the recursive nature of
the problem allows us to rewrite it more compactly in Bellman equation form,
\begin{eqnarray*}
\VFunc_{t}(\mLevBF_{t},\pLevBF_{t}) & = & \max_{\cLevBF_{t}}~\left\{\util(\cLevBF_{t})+\DiscFac \Ex_{t}\left[ \VFunc_{t+1}((\mLevBF_{t}-\cLevBF_{t})\Rfree+ \pLevBF_{t+1}\tShkAll_{t+1},\pLevBF_{t} \PGro  \pShk_{t+1})\right]\right\}
.
\end{eqnarray*}
\end{comment}

\hypertarget{Financial Intermediary}{}
\subsection{Financial Intermediary}

\label{subsec:Financial Intermediary}

The financial intermediary in our model performs a mutual fund activity where it  collects assets from households $A_{t}$ and invests them into government bonds $B_{t}$, stocks $v_{jt}$, and nominal reserves at the central bank $M_{t}$.\\ 

In particular, at the end of period $t$, the assets collected from households $A_{t}$ must be invested into shares $\mathit{v}_{jt}$ of firm $j$ at price  $q^{s}_{jt}$ , government bonds $B_{t}$ at price $q^{b}_{t}$ and nominal reserves $M_{t}$. 

$$A_{t} = \frac{M_{t}}{P_{t}} +q^{b}_{t} B_{t} + \int_{0}^{1} q^{s}_{jt}\mathit{v}_{jt}\,dj$$

where $A_{t} = \int_{0}^{1} a_{it} \, di$ \\

The mutual fund's return in the next period is then 

$$(1+r^{a}_{t+1})  = \frac{  B_{t} + \int_{0}^{1} (q^{s}_{jt+1}+ D_{jt+1})\mathit{v}_{jt} \, dj +(1+i_{t}) \frac{M_{t}}{P_{t+1}}}{A_{t}}$$\\ 

where  $D_{jt+1}$ are dividends of firm $j$ and $i_{t}$ is the nominal interest rate. \\ \\

The mutual fund is risk neutral and looks to maximize its expected return 


$$\max_{\{B_{t}, M_{t} , \mathit{v}_{jt} \}} \mathrm{E}_{t}\left[1+r^{a}_{t+1} \right] = \mathrm{E}\left[ \frac{ B_{t} + \int_{0}^{1} (q^{s}_{jt+1}+ D_{jt+1})\mathit{v}_{jt} \, dj +(1+i_{t}) \frac{M_{t}}{P_{t+1}}}{\frac{M_{t}}{P_{t}} +q^{b}_{t} B_{t} + \int_{0}^{1} q^{s}_{jt}\mathit{v}_{jt}\,dj} \right]$$ \\

 
The first order conditions lead to the no arbitrage equations:

$$ \mathrm{E}_{t}\left[1+r^{a}_{t+1}\right]= \frac{1}{q^{b}_{t}}  =\frac{\mathrm{E}_{t}\left[q^{s}_{jt+1} + D_{jt+1} \right]}{q^{s}_{jt}} = (1+i_{t}) \mathrm{E}_{t}\left[\frac{P_{t}}{P_{t+1}}\right] \equiv 1 +r_{t}$$

where $r_{t}$ is defined to be the real interest rate in period $t$. 

\hypertarget{Firms}{}
\subsection{Firms}

\hypertarget{Final Good Producer}{}
\subsubsection{Final Good Producer}

Perfectly Competitive Final Good Producer aggregates goods with CES technology

$$ Y_{t} = \left(\int_{0}^{1} Y_{jt}^{\frac{\epsilon_{p}-1}{\epsilon_{p}}}\, dj\right)^{\frac{\epsilon_{p}}{\epsilon_{p}-1}}$$

Final Good Producer Profit maximzation Problem

$$ \max_{Y_{jt}} P_{t} \left(\int_{0}^{1} Y_{jt}^{\frac{\epsilon_{p}-1}{\epsilon_{p}}}\, dj\right)^{\frac{\epsilon_{p}}{\epsilon_{p}-1}} - \int_{0}^{1} P_{jt} Y_{jt} ,\ dj $$


This leads to demand

$$ Y_{jt} = \left(\frac {P_{jt}}{P_{t}}\right)^{- \epsilon_{p}} Y_{t}$$

Plugging this demand into $ P_{t}Y_{t} = \int_{0}^{1} P_{jt} Y_{jt} ,\ dj$ , we obtain

Price index $$P_{t} = \left(\int_{0}^{1} P_{jt}^{1-\epsilon_{p}}\,dj \right )^{\frac{1}{1-\epsilon_{p}}}$$


\hypertarget{Intermediate Good Producer}{}
\subsubsection{Intermediate Good Producer}

$$Y_{jt} =  Z_{t}  N_{jt}$$ 

where $log(Z_{t}) = \rho_{Z} log( Z_{t-1}) + \epsilon_{Z}$


 Firm Maximization Problem
 
 Firm $j$ chooses $P_{jt}$ to maximize its dividend $D_{jt}$ and its stock price $q^{s}_{jt} $
 
 $$\max_{\{P_{jt}\}} \overbrace{\frac{(P_{jt} - MC_{t})Y_{jt}}{P_{t}}}^{=D_{jt}} + q^{s}_{jt}\left(P_{jt}\right) $$
 
Given $q^{s}_{jt}\left(P_{jt}\right) = \frac{\mathrm{E}_{t}\left[q^{s}_{jt+1} +D_{jt+1}\left(P_{jt}\right)\right]}{1+r_{t}}$, this is equivalent to: 
 
 $$\max_{\{P_{jt}\}} \mathrm{E}_{t}\left[\sum_{s=0}^{\infty} (\lambda_{P}) ^{s} M_{t,t+s} \left[ \frac{(P_{jt} - MC_{t+s})Y_{jt+s}}{P_{t+s}}\right]\right]$$
 
subject to $$Y_{jt} = \left(\frac {P_{jt}}{P_{t}}\right)^{- \epsilon_{p}} Y_{t}$$
 
where $ \lambda_{P}$ is the  probability a firm cannot change its price,  $M_{t, t+s} = \prod_{k=t}^{t+s-1} \frac{1}{1+r_{k}}$ is the stochastic discount factor and $MC_{t} = \frac{W_{t}}{A_{t}}$ is the marginal cost

Phillips Curve

$$ \pi_{t} = \frac{\mathrm{E}_{t}[\pi_{t+1}]}{1+r^{*}} + \lambda (\mu_{t}-\mu)$$

where $r^{*}$ is the natural rate of interest in the steady state, $\lambda = \frac{(1-\lambda_{p})(1-\frac{\lambda_{p}}{1+r^{*}})}{\lambda_{p}}$,  $ \mu_{t} = log(P_{t}) - log(W_{t}) + log(Z_{t})$ and $\mu = \frac{\epsilon_{p}}{1-\epsilon_{p}}$

\hypertarget{Labor Market}{}
\subsection{Labor Market}


Every worker $i \in [\mho,1]$ provides $n_{igt}$ hours of work to labor union $g \in [0,1]$ and assume $n_{igt} = \mathit{n}_{gt}$. This assumption will imply labor income heterogeneity is only due to transitory income shocks.

Therefore, 

$$n_{it} = \int_{0}^{1} n_{igt}\,dg$$ WLOG assume employed workers are $i \in [\mho,1]$

and 

$$N_{gt} = \int_{\mho}^{1} n_{igt}\,di = (1-\mho_{t}) \mathit{n}_{gt}$$ 

Can think of LHS as labor demand for labor type $g$ o, and RHS is effective labor supply of labor type $g$ from households. There is an underlying assumption that unemployed households supply  "useless" labor. That they're labor supply is effectively useless. Or assumption can be thought as labor union only asks for labor from unemployed households. So if labor unions need to supply $N_{gt}$ to the labor packer, then the labor union will demand $ n_{gt} = \frac{N_{gt}}{1-\mho}$ from each household that is working.


\hypertarget{Competitive Labor Packer}{}
\subsubsection{Competitive Labor Packer}



Perfectly Competitive Labor Packer purchases labor from Labor Unions and  aggregates Labor using CES technology and sells $N_{t}$ to firms at price $W_{t}$


$$ N_{t} = \left(\int_{0}^{1} N_{gt}^{\frac{\epsilon_{w}-1}{\epsilon_{w}}}\,dg\right)^{\frac{\epsilon_{w}}{\epsilon_{w}-1}}$$

The Competitve Labor Packer's profit maximizing Problem 

$$ \max_{n_{jgt}} W_{t} \left(\int_{0}^{1} N_{jgt}^{\frac{\epsilon_{w}-1}{\epsilon_{w}}} \, dg \right)^ {\frac{\epsilon_{w}}{\epsilon_{w}-1}} - \int_{0}^{1} W_{gt}N_{jgt}\, dj $$


 Competitive Labor Packer Demand for labor types

$$ N_{gt} = \left(\frac{W_{gt}}{W_{t}}\right)^{-\epsilon_{w}} N_{t} $$

Wage index follows
$$ W_{t} = \left(\int_{0}^{1} W_{gt}^{1-\epsilon_{w}}\,dg\right)^{\frac{1}{1-\epsilon_{w}}}$$




\hypertarget{Labor Unions}{}
\subsubsection{Labor Unions}

Labor Union Maximization Problem

Labor Union $g$ will set a wage to maximize expected lifetime utility. It may only adjust its price given it is chosen by the calvo fairy. 

$$ \max_{\{W_{gt}\}} \mathrm{E_{t}}\left[\sum_{s=0}^{\infty} (\bar{\beta} \not D \lambda_{w})^{s} \int_{\mho}^{1}  U\left (c_{it+s}(W_{gt+s}), n_{i t+s}) \, di \right)\right] $$

where $\bar{\beta} = \int_{\mho}^{1} \beta_{i} \, di$

subject to the following three constraints $$ N_{gt} = \left(\frac{W_{gt}}{W_{t}}\right)^{-\epsilon_{w}} N_{t} $$

$$ W_{t} = \left(\int_{0}^{1} W_{gt}^{1-\epsilon_{w}}\,dg\right)^{\frac{1}{1-\epsilon_{w}}}$$

where $\lambda_{w}$ probability labor union cannot adjust its wage and $\Phi_{it+s}$ is the distribution of  liquid assets, transitory and permanent shocks over households at period $t+s$. \\


Wage Phillips Curve follows from the first order condition


$$ \pi_{t}^{w} =   \bar{\beta} \not D  \mathrm{E}_{t} \left[ \pi_{t+1}^{w}\right] + \frac{(1-\lambda_{w})}{\lambda_{w}} (1- \bar{\beta} \not D \lambda_{w}) (\mu^{w} - \mu_{t}^{w})$$

where $\mu_{t}^{w} = log\left( \frac{W_{t}}{P_{t}}\right)  - log\left(1 -\tau_{t}\right) - mrs_{t}$


\hypertarget{Government Policy}{}
\subsection{Government Policy}



\hypertarget{Fiscal Policy}{}
\subsubsection{Fiscal Policy}

The government follows the balanced budget

$$ B_{t-1} + G_{t} + \mathit{u} \mho =   q^{b}_{t} B_{t} +  \tau_{t} w_{t} N_{t} $$ 

We will assume $ G_{t} = G$ and $ \tau_{t} = \tau$

\hypertarget{Monetary Policy}{}
\subsubsection{Monetary Policy}


The central bank follows the taylor rule: 

$$i_{t} = r_{t}^{*} +\phi \pi_{t} + \phi_{y} (Y_{t} - Y_{ss}) + v_{t}$$

where $v_{t} = \rho_{v} v_{t-1} +\varepsilon_{t}$


\hypertarget{Equilibrium}{}
\subsection{Equilibrium}


An equilibrium in this economy is a sequence of: \\

- Policy Functions $\left( \mathcal{A}_{t}(m) , \mathcal{C}_{t}(m) \right )_{t=0}^{\infty}$ \\

- Value functions $ \left( V_{t}(m) \right)_{t=0}^{\infty}$\\

- Distributions $ \left(\Phi_{t}(m) \right)_{t=0}^{\infty}$\\

- Prices $ \left( r^{a}_{t}, i_{t}, q^{s}_{t}, q^{b}_{t}, P_{t}, W_{t} , w_{t} , \pi_{t}, \pi^{w}_{t} \right) _{t=0}^{\infty}$\\

- Aggregates $ \left(C_{t}, Y_{t} , N_{t},D_{t} , A_{t} , B_{t} \right)_{t=0}^{\infty}$\\

Such that: \\

$ \left( \mathcal{A}_{t}(m) , \mathcal{C}_{t}(m), V_{t}(m)\right)_{t=0}^{\infty}$  solves the household's maximization problem given $  \left( w_{t}, N_{t},  r^{a}_{t}, \tau_{t}, \Phi_{t-1}(m)\right)_{t=0}^{\infty}$.\\

The Mutual Fund, final Goods producer, intermediate goods producers, labor packer, and labor unions maximize their objective function.

The government budget constraint holds.

The nominal interest rate is set according to the central bank's Taylor rule.


$ \Phi_{t+1}(m) = H(\Phi_{t}(m))$ holds.\\


Markets clear:\\

 $$ A_t =  \int_{0}^{1} \pLevBF_{it}\left( m_{it} - c_{it}(m_{it})\right) \, di $$
 
 $$ Y_t = C_{t} +G $$
 
 where $C_{t} =  \int_{0}^{1} \pLevBF_{it} c_{it}(m_{it})\, di $


\hypertarget{Computation}{}
\section{Computation}

\hypertarget{Calibration}{}
\section{Calibration}

\input{\TableDir/Calibration.tex}






\hypertarget{Results}{}
\section{Results}


\hypertarget{Conclusion}{}
\section{Conclusion}







\providecommand{\figName}{Convergence-of-the-Consumption-Rules}
\providecommand{\figFile}{cFuncsConverge}
\hypertarget{\figFile}{}
\hypertarget{\figName}{}
\begin{figure}[tbp]
\centerline{\includegraphics[width=6in]{\FigDir/\figFile}}
\caption{Convergence of the Consumption Rules}
\label{fig:\figFile}
\end{figure}














\clearpage\vfill\eject

\appendix

\centerline{\LARGE Appendices}\vspace{0.2in}




\clearpage\vfill\eject

\normalsize


\hypertarget{Computational Details}{}
\section{Computational Details}

\hypertarget{Household Bellman Equation }{}
\subsection{Household Bellman Equation}

Household i's dynamic program is

$$ V(\pmb{\mathrm{m}}_{it},\pmb{\mathrm{p}}_{it})=\max_{\{ \pmb{\mathrm{c}}_{it}\}} { U( \pmb{\mathrm{c}}_{it}, n_{it}) + \beta_{i} \not D \mathrm{E}_{t}[V( \pmb{\mathrm{m}}_{it+1} , \pmb{\mathrm{p}}_{it+1})]}$$

subject to 

\begin{align*}
 \pmb{\mathrm{m}}_{i t} & = \pmb{\mathrm{z}}_{i t}  + (1+\mathit{r}^{a}_{t})\pmb{\mathrm{a}}_{i t-1} \\
 \pmb{\mathrm{c}}_{i t}  + \pmb{\mathrm{a}}_{i t} &= \pmb{\mathrm{z}}_{i t}  + (1+\mathit{r}^{a}_{t}) \pmb{\mathrm{a}}_{i t-1}   \\
\pmb{\mathrm{a}}_{it} &\geq 0 
\end{align*} \\ \\

This can be normalized to \\


$$ V(m_{it}) = \max_{\{c_{it}\}} { U(c_{it}, n_{it}) + \beta_{i}\not D \mathrm{E}_{t}[\psi_{it+1}^{1-\rho} V(m_{it+1})]}$$

 subject to 
 
 \begin{align*}
m_{i t} &=  \xi_{it}  + (1+r^{a}_{t}) \frac{a_{i t-1}}{\psi_{it-1}} \\
 c_{i t}  + a_{i t} &= \xi_{it}  + (1+r^{a}_{t}) \frac{a_{i t-1}}{\psi_{it-1}} \\
 a_{it} &\geq 0 
 \end{align*}
 
 Here non boldface variables are normalized by permanent income $\mathit{p_{it}}$. 

e.g. $x_{it} = \frac{\mathbf{x_{it}}}{\pmb{\mathrm{p}}_{it}}$




\hypertarget{Model as System}{}
\subsection{Model as System}

$$
H_{t}(\mathbf{U},\mathbf{Z})= \begin{pmatrix} 
 Y_{t} - Z_{t}N_{t} \\ \\ 
B_{t-1} - q^{b}_{t}B_{t} + u\mho + G_{t} - \tau w_{t} N_{t} \\ \\  
i_{t} - r^{*} - \phi \pi_{t} -\phi_{y}(Y_{t}-Y_{ss}) - v_{t} \\ \\
\pi_{t} -\frac{\pi_{t+1}}{1+r^{*}} + \lambda(\mu_{t}^{p} -\mu_{p})  \\ \\
 \pi_{t}^{w} -\not D \pi_{t+1}^{w} -(\frac{1-\lambda_{w}}{\lambda_{w}}) (1- \not D \lambda_{w}) (\mu^{w} -\mu_{t}^{w}) \\ \\
    1+r_{t} - \frac{1 + i_{t}}{1+ \pi^{p}_{t+1}}\\ \\
 1+r_{t+1}^{a} - \frac{q_{t+1}^{s} +D_{t+1}}{q_{t}^{s}} \\ \\
 r_{t} - r_{t+1}^{a} \\ \\
 \frac{w_{t}}{w_{t-1}} - \frac{\Pi_{t}^{w}}{\Pi_{t}^{p}} \\ \\
 \mathcal{C}_{t}(\{r_{s}^{a} ,w_{s}, N_{s}\}_{s=0}^{s=T}) - Y_{t} + G_{t}  \\ \\
  \mathcal{A}(\{r_{s}^{a} ,w_{s}, N_{s}\}_{s=0}^{s=T}) - q_{t}^{s} - \frac{B_{t}}{1+r_{t}}  \\ \\
 \end{pmatrix} = \begin{pmatrix} 0 \\ 0 \\. \\. \\. \\ 0\\ \end{pmatrix} , \quad t=0,1 ,2,3,....
$$ \\ \\
 

 
 where \\
 
 $\mathcal{C}_{t}(\{r_{s}^{a} ,w_{s}, N_{s}\}_{s=0}^{s=T}) = \int_{0}^{1} \pLevBF_{it} c_{it}(m_{it})\, di $ \\
 
 $\mathcal{A}_{t}(\{r_{s}^{a} ,w_{s}, N_{s}\}_{s=0}^{s=T}) = \int_{0}^{1} \pLevBF_{it} \left(m_{it} - c_{it}(m_{it})\right)\, di $ \\
 
$c_{it}(m_{it})$ is the steady state normalized consumption policy for household $i$ in period t. \\ \\
 

 
 
 $\mathbf{U} = \left(Y_{t} , N_{t} ,  D_{t
 }, B_{t}, w_{t} , \pi_{t}^{p} ,\pi_{t}^{w}, r_{t} , r_{t+1}^{a}, i_{t} , q_{t}^{s},  q_{t}^{s} \right)_{t=0}^{t=T}$ \\ 

 
 $\mathbf{Z} = \left(Z_{t} ,v_{t}\right)_{t=0}^{t=T}$ \\ \\
 
 
\hypertarget{Reduced System}{}
\subsubsection{Reduced System}
 
 System Can be reduced to the following: \\ \\
 
Exogenous Variables are $ Z_{t}, v_{t}$ \\ 

Endogenous Variables are $ r_{t} , w_{t} ,N_{t}$ \\ \\

\begin{eqnarray} 
H_{t}(\mathbf{U},\mathbf{Z})= \begin{pmatrix} 
\mathcal{H}_{t,1} \\ \\ 
\mathcal{H}_{t,2} \\ \\
\mathcal{H}_{t,3} \\ \\
 \end{pmatrix} = \begin{pmatrix} 0 \\ 0 \\ 0 \\ \end{pmatrix} , \quad  t = 0, 1, 2, ..., T 
 \end{eqnarray}
 
 where \\ 
 
 
 $\mathcal{H}_{t,1}  =\mathcal{C}_{t}\left( \left \{r^{a}_{s} , w_{s} , N_{s}  \right \}_{s=0}^{s=T} \right) - Z_{t} N_{t} + G\\ \\ $

$ \mathcal{H}_{t,2}  =log(w_{t}) - log(w_{t-1}) + \left( \frac{1 - \lambda_{w}}{\lambda_{w}}(1 - \not D \lambda_{w}) \sum_{k=0}^{\infty} \not D^{k} ( \mu_{t+k}^{w} - \mu^{w}) \right) - \left(  \lambda \sum_{k=0}^{\infty} \frac{1}{(1+r^{*})^{k}} ( \mu_{t+k}^{p} - \mu^{p})\right)\\ \\ $

$ \mathcal{H}_{t,3}  =  (1+r_{t}) \left(1+ -\lambda \sum_{k=1}^{\infty} \frac{1}{(1+r^{*})^{k}} ( \mu_{t+k}^{p} - \mu^{p}) \right) \\ \\
 - \left(1+r^{*}+ \phi \left(- \lambda \sum_{k=0}^{\infty} \frac{1}{(1+r^{*})^{k}} ( \mu_{t+k}^{p} - \mu^{p}) \right) +\phi_{y} \left(Z_{t} N_{t} - Y_{ss} \right) + v_{t}\right) \\ \\ $
 
$\mathbf{U} = (r_{0} , r_{1} , ...r_{T}, w_{0}, w_{1}, ..., w_{T}, N_{0}, N_{1},...,N_{T})$ \\ 

$ \mathbf{Z} = ( Z_{0}, Z_{1},... Z_{T}, v_{0},...,v_{T}) \\ \\ $ 

$ \mu_{t}^{p} = log(\frac{1}{w_{t}}) + log(Z_{t})$ \\ 

$\mu_{t}^{w} = log(w_{t}) + log(1 - \tau_{t}) - mrs_{t} \\ \\ $


$mrs_{t} = log \left(- \frac{\int_{0}^{1}   U_{n} \left(\cLevBF_{i t}, n_{i t} \right) \ d i  }{\int_{0}^{1} \pLevBF_{it} \theta_{it} U_{c} \left(\cLevBF_{i t}, n_{i t} \right) \  di } \right) = log \left(\frac{\int_{0}^{1} \varphi \pLevBF_{it} n_{it}^{v} \ d i  }{\int_{0}^{1} \pLevBF_{it}  \theta_{it} \cLevBF_{it}^{-\rho} \  di } \right) = log \left(\frac{\int_{0}^{1} \varphi \pLevBF_{it} \left(\frac{N_{t}}{1 - \mho} \right)^{v} \, d i  }{\int_{0}^{1} \pLevBF_{it}  \theta_{it} \cLevBF_{it}^{-\rho} \  di } \right)$ \\ 

$ = log \left( \varphi \left(\frac{N_{t}}{1 - \mho}\right) ^{v}\right) + log \left(\int_{0}^{1} \pLevBF_{it}  \,  di  \right) - log \left(\int_{0}^{1} \pLevBF_{it}  \theta_{it} \cLevBF_{it}^{-\rho} \  di  \right) $


\hypertarget{Jacobian of System}{}
\subsection{Jacobian of System} 

By applying implicit function theorem to equation 10 : \\ \\

$$d\mathbf{U} =  -{\mathbf{H}_{\mathbf{U}}}^{-1} \mathbf{H}_{\mathbf{Z}} d \mathbf{Z}$$ \\ 


 $$  \mathbf{H}_{\mathbf{U}}= \begin{pmatrix} 
H_{\mathbf{u}, 0} \\ \\ 
H_{\mathbf{u}, 1}  \\ \\
. \\ \\
. \\ \\
. \\ \\ 
H_{\mathbf{u}, T} \\ \\
 \end{pmatrix} \quad \quad \mathbf{H}_{\mathbf{Z}}= \begin{pmatrix} 
H_{\mathbf{Z}, 0} \\ \\ 
H_{\mathbf{Z}, 1}  \\ \\
. \\ \\
. \\ \\
. \\ \\ 
H_{\mathbf{Z}, T} \\ \\
 \end{pmatrix}$$ \\ \\
 
 
 
$$ H_{\mathbf{U}, t}= \begin{pmatrix} 
\frac{ \partial \mathcal{H}_{t,1}}{\partial r_{0}}  & ... & \frac{ \partial \mathcal{H}_{t,1}}{\partial r_{T}} & \frac{ \partial \mathcal{H}_{t,1}}{\partial w_{0}} & ... & \frac{ \partial \mathcal{H}_{t,1}}{\partial w_{T}} & \frac{ \partial \mathcal{H}_{t,1}}{\partial N_{0}} & ... &\frac{ \partial \mathcal{H}_{t,1}}{\partial N_{T}} \\ \\ 
\frac{ \partial \mathcal{H}_{t,2}}{\partial r_{0}}  & ... & \frac{ \partial \mathcal{H}_{t,2}}{\partial r_{T}} & \frac{ \partial \mathcal{H}_{t,2}}{\partial w_{0}} & ... & \frac{ \partial \mathcal{H}_{t,2}}{\partial w_{T}} & \frac{ \partial \mathcal{H}_{t,2}}{\partial N_{0}} & ... &\frac{ \partial \mathcal{H}_{t,2}}{\partial N_{T}}  \\ \\
\frac{ \partial \mathcal{H}_{t,3}}{\partial r_{0}}  & ... & \frac{ \partial \mathcal{H}_{t,3}}{\partial r_{T}} & \frac{ \partial \mathcal{H}_{t,3}}{\partial w_{0}} & ... & \frac{ \partial \mathcal{H}_{t,3}}{\partial w_{T}} & \frac{ \partial \mathcal{H}_{t,3}}{\partial N_{0}} & ... &\frac{ \partial \mathcal{H}_{t,3}}{\partial N_{T}} \\ \\
 \end{pmatrix} $$ \\
 
  $$ H_{\mathbf{Z}, t}= \begin{pmatrix} 
\frac{ \partial \mathcal{H}_{t,1}}{\partial Z_{0}}  & ... & \frac{ \partial \mathcal{H}_{t,1}}{\partial Z_{T}} & \frac{ \partial \mathcal{H}_{t,1}}{\partial v_{0}} & ... & \frac{ \partial \mathcal{H}_{t,1}}{\partial v_{T}} \\ \\ 
\frac{ \partial \mathcal{H}_{t,2}}{\partial Z_{0}}  & ... & \frac{ \partial \mathcal{H}_{t,2}}{\partial Z_{T}} & \frac{ \partial \mathcal{H}_{t,2}}{\partial v_{0}} & ... & \frac{ \partial \mathcal{H}_{t,2}}{\partial v_{T}} \\ \\
\frac{ \partial \mathcal{H}_{t,3}}{\partial Z_{0}}  & ... & \frac{ \partial \mathcal{H}_{t,3}}{\partial Z_{T}} & \frac{ \partial \mathcal{H}_{t,3}}{\partial v_{0}} & ... & \frac{ \partial \mathcal{H}_{t,3}}{\partial v_{T}}  \\ \\
 \end{pmatrix} $$
 
 

\bibliography{\econtexRoot/BufferStockTheory,economics}

\end{document}


\provideboolean{Shorter}
\setboolean{Shorter}{true}
\setboolean{Shorter}{false}
\providecommand{\ShorterYN}{\ifthenelse{\boolean{Shorter}}}
\usepackage{rotating}\usepackage{subfigure}


\hypersetup{pdfauthor={William Du <wdu9@jhu.edu>},
            pdftitle={Theoretical Foundations of Buffer Stock Saving},
            pdfkeywords={Precautionary saving, buffer-stock saving, consumption, marginal propensity to consume, permanent income hypothesis},
            pdfcreator = {wdu9@jhu.edu}
}

\begin{document}\bibliographystyle{\econtexBibStyle}
\renewcommand{\onlyinsubfile}[1]{}\renewcommand{\notinsubfile}[1]{#1} 

\hfill{\tiny \texname.tex, \today}

\begin{verbatimwrite}{\texname.title}
Theoretical Foundations of Buffer Stock Saving
\end{verbatimwrite}


\title{Distribution of Wealth and Monetary Policy}

\author{William Du\authNum}

\keywords{Precautionary saving, Heterogeneous Agents, Monetary Policy, permanent income hypothesis}

\jelclass{D81, D91, E21}


\maketitle 


\hypertarget{abstract}{}
\begin{abstract}
  This paper develops a heterogenous Agent New Keynesian Model with a friedman buffer stock income process.
\end{abstract}

\begin{small}
\parbox{\textwidth}{
\begin{center}
\begin{tabbing}
\texttt{~Archive:~} \= \= \url{} \kill \\  %
\texttt{~~~~~PDF:~} \> \> \url{} \\
\texttt{~~Slides:~} \> \> \url{} \\
\texttt{~~~~~Web:~} \> \> \url{}    \\
\texttt{~~GitHub:~} \> \> \url{http://github.com/wdu9/FBS-NK} \\
\texttt{~~~~~~~~~~} \> \> \textit{(In GitHub repo, see \texttt{/Code} for tools for solving and simulating the model)} \\
\end{tabbing}
\end{center}
          
\href{https://colab.research.google.com/github/econ-ark/REMARK/blob/master/REMARKs/BufferStockTheory/BufferStockTheory.ipynb}{CLICK HERE} for an interactive \href{http:https://en.wikipedia.org/wiki/Project_Jupyter}{Jupyter Notebook} that uses the \href{https://econ-ark/HARK}{Econ-ARK/HARK} toolkit (\cite{carroll_et_al-proc-scipy-2018}) to produce all of the paper's figures (warning: it may take several minutes to launch)
}
\end{small}

\begin{authorsinfo}
\name{Contact: \href{mailto:wdu9@jhu.edu}{\texttt{wdu9@jhu.edu}}}
\end{authorsinfo}

\thanks{Thanks to }

\titlepagefinish


\newtheorem{defn}{Definition}
\newtheorem{theorem}{Theorem}

\hypertarget{Introduction}{}
\section{Introduction}

\label{sec:intro}


Write here for intro



\hypertarget{The-Model}{}
\section{The Model}

\subsection{Households}
\label{subsec:Households} 

There is a continuum of households of mass 1 distributed on the unit
interval and indexed by $i$. Households are ex-ante heterogeneous in their discount factors and are subject to idiosyncratic income shocks.  Each household faces the following problem:

\begin{verbatimwrite}{\EqDir/supfn.tex}
\begin{eqnarray}
  \label{eq:supfn}
  \max_{\{\cLevBF_{it+s}\}_{s=0}^{\infty}} \mathrm{E_{t}}\left[\sum_{s=0}^{\infty} (\not D \beta_{i})^{t+s} U\left(  \cLevBF_{i t+s}, n_{i t+s}\right)\right]
\end{eqnarray}
\end{verbatimwrite}
\input{\EqDir/supfn.tex} 

subject to 
\begin{align*}
\aLevBF_{it}     &= \mLevBF_{it} - \cLevBF_{it}   \label{eq:DBCparts} \\
\aLevBF_{it} +\cLevBF_{it}    &= \mathbf{z}_{it} +   (1 + r^{a}_{t} ) \aLevBF_{it-1}   \\ 
\aLevBF_{it}  &\geq 0 \\
\end{align*}

where
$U\left(\cLevBF_{i t}, n_{i t}\right) = \frac{\cLevBF_{i t}^{1-\rho}}{1 -\rho} - \varphi \pLevBF_{it} \frac{n_{it}^{1+v}}{1+v}$ , $\beta_{i}$ is the discount factor of household $i$ and $\not D$ is the probability of death.  \\

$\mLevBF_{it}$ \ denotes household $i$'s market resources at time $t$ to be expended on consumption $\cLevBF_{it}$ or invested into an asset $ \aLevBF_{it}$ with return $r_{t+1}^{a}$.  $\mLevBF_{it}$ is determined by labor income,  $\mathbf{z}_{it}$, and the gross return on assets from the last period, $(1+r_{t}^{a}) \aLevBF_{it-1} $. Labor supply of household $i$ at time $t$ is denoted by $n_{it}$.  Given the formulation of sticky wages described in section 2.4, labor supply is an aggregate state variable and therefore consumption serves as the sole control variable in the dynamic problem. \\




\begin{align*}
\mathbf{z}_{it} &= \pLevBF_{it}\tShkAll_{it} \\
\pLevBF_{it+1} &=\pLevBF_{it} \pShk_{it+1} \\
\end{align*}


Labor income is subject to permanent and transitory idiosyncratic shocks. In particular, household $i$'s labor income is composed of a permanent component, $\pLevBF_{it} $ indicating the level of permanent income and a transitory component, $\tShkAll_{it} $, indicating the transitory income shock received by household $i$ at time $t$. $\pLevBF_{it} $ is subject to permanent income shocks $\pShk_{it+1}$ where $\pShk_{it}$ is iid mean one lognormal with standard deviation $\sigma_\pShk$, $\forall t$  
($\Ex_{t}[{\pShk}_{t+n}]=1~\forall~n>0$) .



The transitory random variable follows   
\begin{verbatimwrite}{\EqDir/tShkDef}
\begin{equation}
\tShkAll _{it+n}=
\begin{cases}
 u \phantom{_{t+1}/\pNotZero} & \text{with probability $\pZero>0$} \\
 \tShkEmp_{it+n} (1-\tau_{t})\int_{0}^{1} w_{gt}n_{igt} \, dg      & \text{with probability $\pNotZero  $} 
\end{cases} \label{eq:tShkDef}
\end{equation}
\end{verbatimwrite}
\input{\EqDir/tShkDef.tex}
where $\tau_{t}$ is the tax rate , $w_{gt}$ is the real wage for labor type $g$ at time t, $ n_{igt}$ is the labor supply for labor type $g$ and $\tShkEmp_{t+n}$ is an iid mean-one lognormal with standard deviation $\sigma_{\tShkEmp}$,
($\Ex_{t}[{\tShkEmp}_{t+n}]=1~\forall~n>0$).




\begin{comment}
Combining the transition equations, the recursive nature of
the problem allows us to rewrite it more compactly in Bellman equation form,
\begin{eqnarray*}
\VFunc_{t}(\mLevBF_{t},\pLevBF_{t}) & = & \max_{\cLevBF_{t}}~\left\{\util(\cLevBF_{t})+\DiscFac \Ex_{t}\left[ \VFunc_{t+1}((\mLevBF_{t}-\cLevBF_{t})\Rfree+ \pLevBF_{t+1}\tShkAll_{t+1},\pLevBF_{t} \PGro  \pShk_{t+1})\right]\right\}
.
\end{eqnarray*}
\end{comment}

\hypertarget{Financial Intermediary}{}
\subsection{Financial Intermediary}

\label{subsec:Financial Intermediary}

The financial intermediary in our model performs a mutual fund activity where it  collects assets from households $A_{t}$ and invests them into government bonds $B_{t}$, stocks $v_{jt}$, and nominal reserves at the central bank $M_{t}$.\\ 

In particular, at the end of period $t$, the assets collected from households $A_{t}$ must be invested into shares $\mathit{v}_{jt}$ of firm $j$ at price  $q^{s}_{jt}$ , government bonds $B_{t}$ at price $q^{b}_{t}$ and nominal reserves $M_{t}$. 

$$A_{t} = \frac{M_{t}}{P_{t}} +q^{b}_{t} B_{t} + \int_{0}^{1} q^{s}_{jt}\mathit{v}_{jt}\,dj$$

where $A_{t} = \int_{0}^{1} a_{it} \, di$ \\

The mutual fund's return in the next period is then 

$$(1+r^{a}_{t+1})  = \frac{  B_{t} + \int_{0}^{1} (q^{s}_{jt+1}+ D_{jt+1})\mathit{v}_{jt} \, dj +(1+i_{t}) \frac{M_{t}}{P_{t+1}}}{A_{t}}$$\\ 

where  $D_{jt+1}$ are dividends of firm $j$ and $i_{t}$ is the nominal interest rate. \\ \\

The mutual fund is risk neutral and looks to maximize its expected return 


$$\max_{\{B_{t}, M_{t} , \mathit{v}_{jt} \}} \mathrm{E}_{t}\left[1+r^{a}_{t+1} \right] = \mathrm{E}\left[ \frac{ B_{t} + \int_{0}^{1} (q^{s}_{jt+1}+ D_{jt+1})\mathit{v}_{jt} \, dj +(1+i_{t}) \frac{M_{t}}{P_{t+1}}}{\frac{M_{t}}{P_{t}} +q^{b}_{t} B_{t} + \int_{0}^{1} q^{s}_{jt}\mathit{v}_{jt}\,dj} \right]$$ \\

 
The first order conditions lead to the no arbitrage equations:

$$ \mathrm{E}_{t}\left[1+r^{a}_{t+1}\right]= \frac{1}{q^{b}_{t}}  =\frac{\mathrm{E}_{t}\left[q^{s}_{jt+1} + D_{jt+1} \right]}{q^{s}_{jt}} = (1+i_{t}) \mathrm{E}_{t}\left[\frac{P_{t}}{P_{t+1}}\right] \equiv 1 +r_{t}$$

where $r_{t}$ is defined to be the real interest rate in period $t$. 

\hypertarget{Firms}{}
\subsection{Firms}

\hypertarget{Final Good Producer}{}
\subsubsection{Final Good Producer}

Perfectly Competitive Final Good Producer aggregates goods with CES technology

$$ Y_{t} = \left(\int_{0}^{1} Y_{jt}^{\frac{\epsilon_{p}-1}{\epsilon_{p}}}\, dj\right)^{\frac{\epsilon_{p}}{\epsilon_{p}-1}}$$

Final Good Producer Profit maximzation Problem

$$ \max_{Y_{jt}} P_{t} \left(\int_{0}^{1} Y_{jt}^{\frac{\epsilon_{p}-1}{\epsilon_{p}}}\, dj\right)^{\frac{\epsilon_{p}}{\epsilon_{p}-1}} - \int_{0}^{1} P_{jt} Y_{jt} ,\ dj $$


This leads to demand

$$ Y_{jt} = \left(\frac {P_{jt}}{P_{t}}\right)^{- \epsilon_{p}} Y_{t}$$

Plugging this demand into $ P_{t}Y_{t} = \int_{0}^{1} P_{jt} Y_{jt} ,\ dj$ , we obtain

Price index $$P_{t} = \left(\int_{0}^{1} P_{jt}^{1-\epsilon_{p}}\,dj \right )^{\frac{1}{1-\epsilon_{p}}}$$


\hypertarget{Intermediate Good Producer}{}
\subsubsection{Intermediate Good Producer}

$$Y_{jt} =  Z_{t}  N_{jt}$$ 

where $log(Z_{t}) = \rho_{Z} log( Z_{t-1}) + \epsilon_{Z}$


 Firm Maximization Problem
 
 Firm $j$ chooses $P_{jt}$ to maximize its dividend $D_{jt}$ and its stock price $q^{s}_{jt} $
 
 $$\max_{\{P_{jt}\}} \overbrace{\frac{(P_{jt} - MC_{t})Y_{jt}}{P_{t}}}^{=D_{jt}} + q^{s}_{jt}\left(P_{jt}\right) $$
 
Given $q^{s}_{jt}\left(P_{jt}\right) = \frac{\mathrm{E}_{t}\left[q^{s}_{jt+1} +D_{jt+1}\left(P_{jt}\right)\right]}{1+r_{t}}$, this is equivalent to: 
 
 $$\max_{\{P_{jt}\}} \mathrm{E}_{t}\left[\sum_{s=0}^{\infty} (\lambda_{P}) ^{s} M_{t,t+s} \left[ \frac{(P_{jt} - MC_{t+s})Y_{jt+s}}{P_{t+s}}\right]\right]$$
 
subject to $$Y_{jt} = \left(\frac {P_{jt}}{P_{t}}\right)^{- \epsilon_{p}} Y_{t}$$
 
where $ \lambda_{P}$ is the  probability a firm cannot change its price,  $M_{t, t+s} = \prod_{k=t}^{t+s-1} \frac{1}{1+r_{k}}$ is the stochastic discount factor and $MC_{t} = \frac{W_{t}}{A_{t}}$ is the marginal cost

Phillips Curve

$$ \pi_{t} = \frac{\mathrm{E}_{t}[\pi_{t+1}]}{1+r^{*}} + \lambda (\mu_{t}-\mu)$$

where $r^{*}$ is the natural rate of interest in the steady state, $\lambda = \frac{(1-\lambda_{p})(1-\frac{\lambda_{p}}{1+r^{*}})}{\lambda_{p}}$,  $ \mu_{t} = log(P_{t}) - log(W_{t}) + log(Z_{t})$ and $\mu = \frac{\epsilon_{p}}{1-\epsilon_{p}}$

\hypertarget{Labor Market}{}
\subsection{Labor Market}


Every worker $i \in [\mho,1]$ provides $n_{igt}$ hours of work to labor union $g \in [0,1]$ and assume $n_{igt} = \mathit{n}_{gt}$. This assumption will imply labor income heterogeneity is only due to transitory income shocks.

Therefore, 

$$n_{it} = \int_{0}^{1} n_{igt}\,dg$$ WLOG assume employed workers are $i \in [\mho,1]$

and 

$$N_{gt} = \int_{\mho}^{1} n_{igt}\,di = (1-\mho_{t}) \mathit{n}_{gt}$$ 

Can think of LHS as labor demand for labor type $g$ o, and RHS is effective labor supply of labor type $g$ from households. There is an underlying assumption that unemployed households supply  "useless" labor. That they're labor supply is effectively useless. Or assumption can be thought as labor union only asks for labor from unemployed households. So if labor unions need to supply $N_{gt}$ to the labor packer, then the labor union will demand $ n_{gt} = \frac{N_{gt}}{1-\mho}$ from each household that is working.


\hypertarget{Competitive Labor Packer}{}
\subsubsection{Competitive Labor Packer}



Perfectly Competitive Labor Packer purchases labor from Labor Unions and  aggregates Labor using CES technology and sells $N_{t}$ to firms at price $W_{t}$


$$ N_{t} = \left(\int_{0}^{1} N_{gt}^{\frac{\epsilon_{w}-1}{\epsilon_{w}}}\,dg\right)^{\frac{\epsilon_{w}}{\epsilon_{w}-1}}$$

The Competitve Labor Packer's profit maximizing Problem 

$$ \max_{n_{jgt}} W_{t} \left(\int_{0}^{1} N_{jgt}^{\frac{\epsilon_{w}-1}{\epsilon_{w}}} \, dg \right)^ {\frac{\epsilon_{w}}{\epsilon_{w}-1}} - \int_{0}^{1} W_{gt}N_{jgt}\, dj $$


 Competitive Labor Packer Demand for labor types

$$ N_{gt} = \left(\frac{W_{gt}}{W_{t}}\right)^{-\epsilon_{w}} N_{t} $$

Wage index follows
$$ W_{t} = \left(\int_{0}^{1} W_{gt}^{1-\epsilon_{w}}\,dg\right)^{\frac{1}{1-\epsilon_{w}}}$$




\hypertarget{Labor Unions}{}
\subsubsection{Labor Unions}

Labor Union Maximization Problem

Labor Union $g$ will set a wage to maximize expected lifetime utility. It may only adjust its price given it is chosen by the calvo fairy. 

$$ \max_{\{W_{gt}\}} \mathrm{E_{t}}\left[\sum_{s=0}^{\infty} (\bar{\beta} \not D \lambda_{w})^{s} \int_{\mho}^{1}  U\left (c_{it+s}(W_{gt+s}), n_{i t+s}) \, di \right)\right] $$

where $\bar{\beta} = \int_{\mho}^{1} \beta_{i} \, di$

subject to the following three constraints $$ N_{gt} = \left(\frac{W_{gt}}{W_{t}}\right)^{-\epsilon_{w}} N_{t} $$

$$ W_{t} = \left(\int_{0}^{1} W_{gt}^{1-\epsilon_{w}}\,dg\right)^{\frac{1}{1-\epsilon_{w}}}$$

where $\lambda_{w}$ probability labor union cannot adjust its wage and $\Phi_{it+s}$ is the distribution of  liquid assets, transitory and permanent shocks over households at period $t+s$. \\


Wage Phillips Curve follows from the first order condition


$$ \pi_{t}^{w} =   \bar{\beta} \not D  \mathrm{E}_{t} \left[ \pi_{t+1}^{w}\right] + \frac{(1-\lambda_{w})}{\lambda_{w}} (1- \bar{\beta} \not D \lambda_{w}) (\mu^{w} - \mu_{t}^{w})$$

where $\mu_{t}^{w} = log\left( \frac{W_{t}}{P_{t}}\right)  - log\left(1 -\tau_{t}\right) - mrs_{t}$


\hypertarget{Government Policy}{}
\subsection{Government Policy}



\hypertarget{Fiscal Policy}{}
\subsubsection{Fiscal Policy}

The government follows the balanced budget

$$ B_{t-1} + G_{t} + \mathit{u} \mho =   q^{b}_{t} B_{t} +  \tau_{t} w_{t} N_{t} $$ 

We will assume $ G_{t} = G$ and $ \tau_{t} = \tau$

\hypertarget{Monetary Policy}{}
\subsubsection{Monetary Policy}


The central bank follows the taylor rule: 

$$i_{t} = r_{t}^{*} +\phi \pi_{t} + \phi_{y} (Y_{t} - Y_{ss}) + v_{t}$$

where $v_{t} = \rho_{v} v_{t-1} +\varepsilon_{t}$


\hypertarget{Equilibrium}{}
\subsection{Equilibrium}


An equilibrium in this economy is a sequence of: \\

- Policy Functions $\left( \mathcal{A}_{t}(m) , \mathcal{C}_{t}(m) \right )_{t=0}^{\infty}$ \\

- Value functions $ \left( V_{t}(m) \right)_{t=0}^{\infty}$\\

- Distributions $ \left(\Phi_{t}(m) \right)_{t=0}^{\infty}$\\

- Prices $ \left( r^{a}_{t}, i_{t}, q^{s}_{t}, q^{b}_{t}, P_{t}, W_{t} , w_{t} , \pi_{t}, \pi^{w}_{t} \right) _{t=0}^{\infty}$\\

- Aggregates $ \left(C_{t}, Y_{t} , N_{t},D_{t} , A_{t} , B_{t} \right)_{t=0}^{\infty}$\\

Such that: \\

$ \left( \mathcal{A}_{t}(m) , \mathcal{C}_{t}(m), V_{t}(m)\right)_{t=0}^{\infty}$  solves the household's maximization problem given $  \left( w_{t}, N_{t},  r^{a}_{t}, \tau_{t}, \Phi_{t-1}(m)\right)_{t=0}^{\infty}$.\\

The Mutual Fund, final Goods producer, intermediate goods producers, labor packer, and labor unions maximize their objective function.

The government budget constraint holds.

The nominal interest rate is set according to the central bank's Taylor rule.


$ \Phi_{t+1}(m) = H(\Phi_{t}(m))$ holds.\\


Markets clear:\\

 $$ A_t =  \int_{0}^{1} \pLevBF_{it}\left( m_{it} - c_{it}(m_{it})\right) \, di $$
 
 $$ Y_t = C_{t} +G $$
 
 where $C_{t} =  \int_{0}^{1} \pLevBF_{it} c_{it}(m_{it})\, di $


\hypertarget{Computation}{}
\section{Computation}

\hypertarget{Calibration}{}
\section{Calibration}

\begin{table}
\begin{center}\renewcommand{\arraystretch}{1.5}
\caption{Household Calibration}\label{table:Calibration}
\begin{tabular}{|c|ccl|c|}
\hline
\multicolumn{5}{|l|}{Calibrated Parameters}  \\ \hline
Description                     & \multicolumn{1}{c}{Parameter} & Value & \multicolumn{2}{c|}{Source/Target }\\ \hline
Coefficient of Relative Risk Aversion & \multicolumn{1}{c}{$\CRRA$} & 2 & \multicolumn{2}{c|}{Conventional} \\
Real Interest Rate                 & \multicolumn{1}{c}{$r$} & $1.048^{.25} - 1$ & \multicolumn{2}{c|}{Conventional} \\
Discount Factor          & \multicolumn{1}{c}{$\beta$} & 0.96 & \multicolumn{2}{c|}{Conventional} \\
Disutility of Labor Coefficient & \multicolumn{1}{c}{$\varphi$} & .883 & \multicolumn{2}{c|}{N = 1.22} \\
Probability of Death       & \multicolumn{1}{c}{$\pZero$} & 0.00625 & \multicolumn{2}{c|}{} \\
Tax Rate       & \multicolumn{1}{c}{$\tau$} & 0.165 & \multicolumn{2}{c|}{} \\
Frisch        & \multicolumn{1}{c}{$\frac{1}{v}$} & .5 & \multicolumn{2}{c|}{Conventional} \\
Unemployment Benefits       & \multicolumn{1}{c}{$u$} & 0.095 & \multicolumn{2}{c|}{} \\
Probability of Unemployment       & \multicolumn{1}{c}{$\mho$} & 0.05 & \multicolumn{2}{c|}{} \\
Std Dev of Log Permanent Shock  & \multicolumn{1}{c}{$\sigma_{\pshk}$} & 0.06 & \multicolumn{2}{c|}{} \\
Std Dev of Log Transitory Shock & \multicolumn{1}{c}{$\sigma_{\theta}$} & 0.2 & \multicolumn{2}{c|}{} \\ \hline
\end{tabular}
\end{center}
\end{table}
\begin{table}
\begin{center}\renewcommand{\arraystretch}{1.5}
\caption{Economy Calibration}\label{table:Calibration}
\begin{tabular}{|c|ccl|c|}
\hline
\multicolumn{5}{|l|}{Calibrated Parameters}  \\ \hline
Description                     & \multicolumn{1}{c}{Parameter} & Value & \multicolumn{2}{c|}{Source/Target }\\ \hline
Calvo Price Stickiness & \multicolumn{1}{c}{$\Lambda_{p}$} & .8 & \multicolumn{2}{c|}{Conventional} \\
Calvo Wage Stickiness                & \multicolumn{1}{c}{$\Lambda_{w}$} & $.7$ & \multicolumn{2}{c|}{} \\
Steady State Price Markup          & \multicolumn{1}{c}{$\mu_{p}$} & 1.012 & \multicolumn{2}{c|}{} \\
Steady State Price Markup          & \multicolumn{1}{c}{$\mu_{w}$} & 1.05 & \multicolumn{2}{c|}{} \\
 Government Spending       & \multicolumn{1}{c}{$G$} & 0.19 & \multicolumn{2}{c|}{} \\
 Steady State Bond Supply       & \multicolumn{1}{c}{$B$} & 0.5 & \multicolumn{2}{c|}{} \\
Taylor Rule Inflation Coefficient        & \multicolumn{1}{c}{$\phi_{\pi}$} & .8 & \multicolumn{2}{c|}{Conventional} \\
 Taylor Rule Output Gap Coefficient       & \multicolumn{1}{c}{$\phi_{y}$} & 0 & \multicolumn{2}{c|}{} \\
Assets to Output Ratio       & \multicolumn{1}{c}{$\frac{A}{Y}$} & 1.4 & \multicolumn{2}{c|}{} \\
Government Bond to Output Ratio & \multicolumn{1}{c}{$\frac{B}{Y}$} & 0.4 & \multicolumn{2}{c|}{} \\ \hline
\end{tabular}
\end{center}
\end{table}






\hypertarget{Results}{}
\section{Results}


\hypertarget{Conclusion}{}
\section{Conclusion}







\providecommand{\figName}{Convergence-of-the-Consumption-Rules}
\providecommand{\figFile}{cFuncsConverge}
\hypertarget{\figFile}{}
\hypertarget{\figName}{}
\begin{figure}[tbp]
\centerline{\includegraphics[width=6in]{\FigDir/\figFile}}
\caption{Convergence of the Consumption Rules}
\label{fig:\figFile}
\end{figure}














\clearpage\vfill\eject

\appendix

\centerline{\LARGE Appendices}\vspace{0.2in}




\clearpage\vfill\eject

\normalsize


\hypertarget{Computational Details}{}
\section{Computational Details}

\hypertarget{Household Bellman Equation }{}
\subsection{Household Bellman Equation}

Household i's dynamic program is

$$ V(\pmb{\mathrm{m}}_{it},\pmb{\mathrm{p}}_{it})=\max_{\{ \pmb{\mathrm{c}}_{it}\}} { U( \pmb{\mathrm{c}}_{it}, n_{it}) + \beta_{i} \not D \mathrm{E}_{t}[V( \pmb{\mathrm{m}}_{it+1} , \pmb{\mathrm{p}}_{it+1})]}$$

subject to 

\begin{align*}
 \pmb{\mathrm{m}}_{i t} & = \pmb{\mathrm{z}}_{i t}  + (1+\mathit{r}^{a}_{t})\pmb{\mathrm{a}}_{i t-1} \\
 \pmb{\mathrm{c}}_{i t}  + \pmb{\mathrm{a}}_{i t} &= \pmb{\mathrm{z}}_{i t}  + (1+\mathit{r}^{a}_{t}) \pmb{\mathrm{a}}_{i t-1}   \\
\pmb{\mathrm{a}}_{it} &\geq 0 
\end{align*} \\ \\

This can be normalized to \\


$$ V(m_{it}) = \max_{\{c_{it}\}} { U(c_{it}, n_{it}) + \beta_{i}\not D \mathrm{E}_{t}[\psi_{it+1}^{1-\rho} V(m_{it+1})]}$$

 subject to 
 
 \begin{align*}
m_{i t} &=  \xi_{it}  + (1+r^{a}_{t}) \frac{a_{i t-1}}{\psi_{it-1}} \\
 c_{i t}  + a_{i t} &= \xi_{it}  + (1+r^{a}_{t}) \frac{a_{i t-1}}{\psi_{it-1}} \\
 a_{it} &\geq 0 
 \end{align*}
 
 Here non boldface variables are normalized by permanent income $\mathit{p_{it}}$. 

e.g. $x_{it} = \frac{\mathbf{x_{it}}}{\pmb{\mathrm{p}}_{it}}$




\hypertarget{Model as System}{}
\subsection{Model as System}

$$
H_{t}(\mathbf{U},\mathbf{Z})= \begin{pmatrix} 
 Y_{t} - Z_{t}N_{t} \\ \\ 
B_{t-1} - q^{b}_{t}B_{t} + u\mho + G_{t} - \tau w_{t} N_{t} \\ \\  
i_{t} - r^{*} - \phi \pi_{t} -\phi_{y}(Y_{t}-Y_{ss}) - v_{t} \\ \\
\pi_{t} -\frac{\pi_{t+1}}{1+r^{*}} + \lambda(\mu_{t}^{p} -\mu_{p})  \\ \\
 \pi_{t}^{w} -\not D \pi_{t+1}^{w} -(\frac{1-\lambda_{w}}{\lambda_{w}}) (1- \not D \lambda_{w}) (\mu^{w} -\mu_{t}^{w}) \\ \\
    1+r_{t} - \frac{1 + i_{t}}{1+ \pi^{p}_{t+1}}\\ \\
 1+r_{t+1}^{a} - \frac{q_{t+1}^{s} +D_{t+1}}{q_{t}^{s}} \\ \\
 r_{t} - r_{t+1}^{a} \\ \\
 \frac{w_{t}}{w_{t-1}} - \frac{\Pi_{t}^{w}}{\Pi_{t}^{p}} \\ \\
 \mathcal{C}_{t}(\{r_{s}^{a} ,w_{s}, N_{s}\}_{s=0}^{s=T}) - Y_{t} + G_{t}  \\ \\
  \mathcal{A}(\{r_{s}^{a} ,w_{s}, N_{s}\}_{s=0}^{s=T}) - q_{t}^{s} - \frac{B_{t}}{1+r_{t}}  \\ \\
 \end{pmatrix} = \begin{pmatrix} 0 \\ 0 \\. \\. \\. \\ 0\\ \end{pmatrix} , \quad t=0,1 ,2,3,....
$$ \\ \\
 

 
 where \\
 
 $\mathcal{C}_{t}(\{r_{s}^{a} ,w_{s}, N_{s}\}_{s=0}^{s=T}) = \int_{0}^{1} \pLevBF_{it} c_{it}(m_{it})\, di $ \\
 
 $\mathcal{A}_{t}(\{r_{s}^{a} ,w_{s}, N_{s}\}_{s=0}^{s=T}) = \int_{0}^{1} \pLevBF_{it} \left(m_{it} - c_{it}(m_{it})\right)\, di $ \\
 
$c_{it}(m_{it})$ is the steady state normalized consumption policy for household $i$ in period t. \\ \\
 

 
 
 $\mathbf{U} = \left(Y_{t} , N_{t} ,  D_{t
 }, B_{t}, w_{t} , \pi_{t}^{p} ,\pi_{t}^{w}, r_{t} , r_{t+1}^{a}, i_{t} , q_{t}^{s},  q_{t}^{s} \right)_{t=0}^{t=T}$ \\ 

 
 $\mathbf{Z} = \left(Z_{t} ,v_{t}\right)_{t=0}^{t=T}$ \\ \\
 
 
\hypertarget{Reduced System}{}
\subsubsection{Reduced System}
 
 System Can be reduced to the following: \\ \\
 
Exogenous Variables are $ Z_{t}, v_{t}$ \\ 

Endogenous Variables are $ r_{t} , w_{t} ,N_{t}$ \\ \\

\begin{eqnarray} 
H_{t}(\mathbf{U},\mathbf{Z})= \begin{pmatrix} 
\mathcal{H}_{t,1} \\ \\ 
\mathcal{H}_{t,2} \\ \\
\mathcal{H}_{t,3} \\ \\
 \end{pmatrix} = \begin{pmatrix} 0 \\ 0 \\ 0 \\ \end{pmatrix} , \quad  t = 0, 1, 2, ..., T 
 \end{eqnarray}
 
 where \\ 
 
 
 $\mathcal{H}_{t,1}  =\mathcal{C}_{t}\left( \left \{r^{a}_{s} , w_{s} , N_{s}  \right \}_{s=0}^{s=T} \right) - Z_{t} N_{t} + G\\ \\ $

$ \mathcal{H}_{t,2}  =log(w_{t}) - log(w_{t-1}) + \left( \frac{1 - \lambda_{w}}{\lambda_{w}}(1 - \not D \lambda_{w}) \sum_{k=0}^{\infty} \not D^{k} ( \mu_{t+k}^{w} - \mu^{w}) \right) - \left(  \lambda \sum_{k=0}^{\infty} \frac{1}{(1+r^{*})^{k}} ( \mu_{t+k}^{p} - \mu^{p})\right)\\ \\ $

$ \mathcal{H}_{t,3}  =  (1+r_{t}) \left(1+ -\lambda \sum_{k=1}^{\infty} \frac{1}{(1+r^{*})^{k}} ( \mu_{t+k}^{p} - \mu^{p}) \right) \\ \\
 - \left(1+r^{*}+ \phi \left(- \lambda \sum_{k=0}^{\infty} \frac{1}{(1+r^{*})^{k}} ( \mu_{t+k}^{p} - \mu^{p}) \right) +\phi_{y} \left(Z_{t} N_{t} - Y_{ss} \right) + v_{t}\right) \\ \\ $
 
$\mathbf{U} = (r_{0} , r_{1} , ...r_{T}, w_{0}, w_{1}, ..., w_{T}, N_{0}, N_{1},...,N_{T})$ \\ 

$ \mathbf{Z} = ( Z_{0}, Z_{1},... Z_{T}, v_{0},...,v_{T}) \\ \\ $ 

$ \mu_{t}^{p} = log(\frac{1}{w_{t}}) + log(Z_{t})$ \\ 

$\mu_{t}^{w} = log(w_{t}) + log(1 - \tau_{t}) - mrs_{t} \\ \\ $


$mrs_{t} = log \left(- \frac{\int_{0}^{1}   U_{n} \left(\cLevBF_{i t}, n_{i t} \right) \ d i  }{\int_{0}^{1} \pLevBF_{it} \theta_{it} U_{c} \left(\cLevBF_{i t}, n_{i t} \right) \  di } \right) = log \left(\frac{\int_{0}^{1} \varphi \pLevBF_{it} n_{it}^{v} \ d i  }{\int_{0}^{1} \pLevBF_{it}  \theta_{it} \cLevBF_{it}^{-\rho} \  di } \right) = log \left(\frac{\int_{0}^{1} \varphi \pLevBF_{it} \left(\frac{N_{t}}{1 - \mho} \right)^{v} \, d i  }{\int_{0}^{1} \pLevBF_{it}  \theta_{it} \cLevBF_{it}^{-\rho} \  di } \right)$ \\ 

$ = log \left( \varphi \left(\frac{N_{t}}{1 - \mho}\right) ^{v}\right) + log \left(\int_{0}^{1} \pLevBF_{it}  \,  di  \right) - log \left(\int_{0}^{1} \pLevBF_{it}  \theta_{it} \cLevBF_{it}^{-\rho} \  di  \right) $


\hypertarget{Jacobian of System}{}
\subsection{Jacobian of System} 

By applying implicit function theorem to equation 10 : \\ \\

$$d\mathbf{U} =  -{\mathbf{H}_{\mathbf{U}}}^{-1} \mathbf{H}_{\mathbf{Z}} d \mathbf{Z}$$ \\ 


 $$  \mathbf{H}_{\mathbf{U}}= \begin{pmatrix} 
H_{\mathbf{u}, 0} \\ \\ 
H_{\mathbf{u}, 1}  \\ \\
. \\ \\
. \\ \\
. \\ \\ 
H_{\mathbf{u}, T} \\ \\
 \end{pmatrix} \quad \quad \mathbf{H}_{\mathbf{Z}}= \begin{pmatrix} 
H_{\mathbf{Z}, 0} \\ \\ 
H_{\mathbf{Z}, 1}  \\ \\
. \\ \\
. \\ \\
. \\ \\ 
H_{\mathbf{Z}, T} \\ \\
 \end{pmatrix}$$ \\ \\
 
 
 
$$ H_{\mathbf{U}, t}= \begin{pmatrix} 
\frac{ \partial \mathcal{H}_{t,1}}{\partial r_{0}}  & ... & \frac{ \partial \mathcal{H}_{t,1}}{\partial r_{T}} & \frac{ \partial \mathcal{H}_{t,1}}{\partial w_{0}} & ... & \frac{ \partial \mathcal{H}_{t,1}}{\partial w_{T}} & \frac{ \partial \mathcal{H}_{t,1}}{\partial N_{0}} & ... &\frac{ \partial \mathcal{H}_{t,1}}{\partial N_{T}} \\ \\ 
\frac{ \partial \mathcal{H}_{t,2}}{\partial r_{0}}  & ... & \frac{ \partial \mathcal{H}_{t,2}}{\partial r_{T}} & \frac{ \partial \mathcal{H}_{t,2}}{\partial w_{0}} & ... & \frac{ \partial \mathcal{H}_{t,2}}{\partial w_{T}} & \frac{ \partial \mathcal{H}_{t,2}}{\partial N_{0}} & ... &\frac{ \partial \mathcal{H}_{t,2}}{\partial N_{T}}  \\ \\
\frac{ \partial \mathcal{H}_{t,3}}{\partial r_{0}}  & ... & \frac{ \partial \mathcal{H}_{t,3}}{\partial r_{T}} & \frac{ \partial \mathcal{H}_{t,3}}{\partial w_{0}} & ... & \frac{ \partial \mathcal{H}_{t,3}}{\partial w_{T}} & \frac{ \partial \mathcal{H}_{t,3}}{\partial N_{0}} & ... &\frac{ \partial \mathcal{H}_{t,3}}{\partial N_{T}} \\ \\
 \end{pmatrix} $$ \\
 
  $$ H_{\mathbf{Z}, t}= \begin{pmatrix} 
\frac{ \partial \mathcal{H}_{t,1}}{\partial Z_{0}}  & ... & \frac{ \partial \mathcal{H}_{t,1}}{\partial Z_{T}} & \frac{ \partial \mathcal{H}_{t,1}}{\partial v_{0}} & ... & \frac{ \partial \mathcal{H}_{t,1}}{\partial v_{T}} \\ \\ 
\frac{ \partial \mathcal{H}_{t,2}}{\partial Z_{0}}  & ... & \frac{ \partial \mathcal{H}_{t,2}}{\partial Z_{T}} & \frac{ \partial \mathcal{H}_{t,2}}{\partial v_{0}} & ... & \frac{ \partial \mathcal{H}_{t,2}}{\partial v_{T}} \\ \\
\frac{ \partial \mathcal{H}_{t,3}}{\partial Z_{0}}  & ... & \frac{ \partial \mathcal{H}_{t,3}}{\partial Z_{T}} & \frac{ \partial \mathcal{H}_{t,3}}{\partial v_{0}} & ... & \frac{ \partial \mathcal{H}_{t,3}}{\partial v_{T}}  \\ \\
 \end{pmatrix} $$
 
 

\bibliography{\econtexRoot/BufferStockTheory,economics}

\end{document}


\provideboolean{Shorter}
\setboolean{Shorter}{true}
\setboolean{Shorter}{false}
\providecommand{\ShorterYN}{\ifthenelse{\boolean{Shorter}}}
\usepackage{rotating}\usepackage{subfigure}


\hypersetup{pdfauthor={William Du <wdu9@jhu.edu>},
            pdftitle={Theoretical Foundations of Buffer Stock Saving},
            pdfkeywords={Precautionary saving, buffer-stock saving, consumption, marginal propensity to consume, permanent income hypothesis},
            pdfcreator = {wdu9@jhu.edu}
}

\begin{document}\bibliographystyle{\econtexBibStyle}
\renewcommand{\onlyinsubfile}[1]{}\renewcommand{\notinsubfile}[1]{#1} 

\hfill{\tiny \texname.tex, \today}

\begin{verbatimwrite}{\texname.title}
Theoretical Foundations of Buffer Stock Saving
\end{verbatimwrite}


\title{Distribution of Wealth and Monetary Policy}

\author{William Du\authNum}

\keywords{Precautionary saving, Heterogeneous Agents, Monetary Policy, permanent income hypothesis}

\jelclass{D81, D91, E21}


\maketitle 


\hypertarget{abstract}{}
\begin{abstract}
  This paper develops a heterogenous Agent New Keynesian Model with a friedman buffer stock income process.
\end{abstract}

\begin{small}
\parbox{\textwidth}{
\begin{center}
\begin{tabbing}
\texttt{~Archive:~} \= \= \url{} \kill \\  %
\texttt{~~~~~PDF:~} \> \> \url{} \\
\texttt{~~Slides:~} \> \> \url{} \\
\texttt{~~~~~Web:~} \> \> \url{}    \\
\texttt{~~GitHub:~} \> \> \url{http://github.com/wdu9/FBS-NK} \\
\texttt{~~~~~~~~~~} \> \> \textit{(In GitHub repo, see \texttt{/Code} for tools for solving and simulating the model)} \\
\end{tabbing}
\end{center}
          
\href{https://colab.research.google.com/github/econ-ark/REMARK/blob/master/REMARKs/BufferStockTheory/BufferStockTheory.ipynb}{CLICK HERE} for an interactive \href{http:https://en.wikipedia.org/wiki/Project_Jupyter}{Jupyter Notebook} that uses the \href{https://econ-ark/HARK}{Econ-ARK/HARK} toolkit (\cite{carroll_et_al-proc-scipy-2018}) to produce all of the paper's figures (warning: it may take several minutes to launch)
}
\end{small}

\begin{authorsinfo}
\name{Contact: \href{mailto:wdu9@jhu.edu}{\texttt{wdu9@jhu.edu}}}
\end{authorsinfo}

\thanks{Thanks to }

\titlepagefinish


\newtheorem{defn}{Definition}
\newtheorem{theorem}{Theorem}

\hypertarget{Introduction}{}
\section{Introduction}

\label{sec:intro}


Write here for intro



\hypertarget{The-Model}{}
\section{The Model}

\subsection{Households}
\label{subsec:Households} 

There is a continuum of households of mass 1 distributed on the unit
interval and indexed by $i$. Households are ex-ante heterogeneous in their discount factors and are subject to idiosyncratic income shocks.  Each household faces the following problem:

\begin{verbatimwrite}{\EqDir/supfn.tex}
\begin{eqnarray}
  \label{eq:supfn}
  \max_{\{\cLevBF_{it+s}\}_{s=0}^{\infty}} \mathrm{E_{t}}\left[\sum_{s=0}^{\infty} (\not D \beta_{i})^{t+s} U\left(  \cLevBF_{i t+s}, n_{i t+s}\right)\right]
\end{eqnarray}
\end{verbatimwrite}
\input{\EqDir/supfn.tex} 

subject to 
\begin{align*}
\aLevBF_{it}     &= \mLevBF_{it} - \cLevBF_{it}   \label{eq:DBCparts} \\
\aLevBF_{it} +\cLevBF_{it}    &= \mathbf{z}_{it} +   (1 + r^{a}_{t} ) \aLevBF_{it-1}   \\ 
\aLevBF_{it}  &\geq 0 \\
\end{align*}

where
$U\left(\cLevBF_{i t}, n_{i t}\right) = \frac{\cLevBF_{i t}^{1-\rho}}{1 -\rho} - \varphi \pLevBF_{it} \frac{n_{it}^{1+v}}{1+v}$ , $\beta_{i}$ is the discount factor of household $i$ and $\not D$ is the probability of death.  \\

$\mLevBF_{it}$ \ denotes household $i$'s market resources at time $t$ to be expended on consumption $\cLevBF_{it}$ or invested into an asset $ \aLevBF_{it}$ with return $r_{t+1}^{a}$.  $\mLevBF_{it}$ is determined by labor income,  $\mathbf{z}_{it}$, and the gross return on assets from the last period, $(1+r_{t}^{a}) \aLevBF_{it-1} $. Labor supply of household $i$ at time $t$ is denoted by $n_{it}$.  Given the formulation of sticky wages described in section 2.4, labor supply is an aggregate state variable and therefore consumption serves as the sole control variable in the dynamic problem. \\




\begin{align*}
\mathbf{z}_{it} &= \pLevBF_{it}\tShkAll_{it} \\
\pLevBF_{it+1} &=\pLevBF_{it} \pShk_{it+1} \\
\end{align*}


Labor income is subject to permanent and transitory idiosyncratic shocks. In particular, household $i$'s labor income is composed of a permanent component, $\pLevBF_{it} $ indicating the level of permanent income and a transitory component, $\tShkAll_{it} $, indicating the transitory income shock received by household $i$ at time $t$. $\pLevBF_{it} $ is subject to permanent income shocks $\pShk_{it+1}$ where $\pShk_{it}$ is iid mean one lognormal with standard deviation $\sigma_\pShk$, $\forall t$  
($\Ex_{t}[{\pShk}_{t+n}]=1~\forall~n>0$) .



The transitory random variable follows   
\begin{verbatimwrite}{\EqDir/tShkDef}
\begin{equation}
\tShkAll _{it+n}=
\begin{cases}
 u \phantom{_{t+1}/\pNotZero} & \text{with probability $\pZero>0$} \\
 \tShkEmp_{it+n} (1-\tau_{t})\int_{0}^{1} w_{gt}n_{igt} \, dg      & \text{with probability $\pNotZero  $} 
\end{cases} \label{eq:tShkDef}
\end{equation}
\end{verbatimwrite}
\input{\EqDir/tShkDef.tex}
where $\tau_{t}$ is the tax rate , $w_{gt}$ is the real wage for labor type $g$ at time t, $ n_{igt}$ is the labor supply for labor type $g$ and $\tShkEmp_{t+n}$ is an iid mean-one lognormal with standard deviation $\sigma_{\tShkEmp}$,
($\Ex_{t}[{\tShkEmp}_{t+n}]=1~\forall~n>0$).




\begin{comment}
Combining the transition equations, the recursive nature of
the problem allows us to rewrite it more compactly in Bellman equation form,
\begin{eqnarray*}
\VFunc_{t}(\mLevBF_{t},\pLevBF_{t}) & = & \max_{\cLevBF_{t}}~\left\{\util(\cLevBF_{t})+\DiscFac \Ex_{t}\left[ \VFunc_{t+1}((\mLevBF_{t}-\cLevBF_{t})\Rfree+ \pLevBF_{t+1}\tShkAll_{t+1},\pLevBF_{t} \PGro  \pShk_{t+1})\right]\right\}
.
\end{eqnarray*}
\end{comment}

\hypertarget{Financial Intermediary}{}
\subsection{Financial Intermediary}

\label{subsec:Financial Intermediary}

The financial intermediary in our model performs a mutual fund activity where it  collects assets from households $A_{t}$ and invests them into government bonds $B_{t}$, stocks $v_{jt}$, and nominal reserves at the central bank $M_{t}$.\\ 

In particular, at the end of period $t$, the assets collected from households $A_{t}$ must be invested into shares $\mathit{v}_{jt}$ of firm $j$ at price  $q^{s}_{jt}$ , government bonds $B_{t}$ at price $q^{b}_{t}$ and nominal reserves $M_{t}$. 

$$A_{t} = \frac{M_{t}}{P_{t}} +q^{b}_{t} B_{t} + \int_{0}^{1} q^{s}_{jt}\mathit{v}_{jt}\,dj$$

where $A_{t} = \int_{0}^{1} a_{it} \, di$ \\

The mutual fund's return in the next period is then 

$$(1+r^{a}_{t+1})  = \frac{  B_{t} + \int_{0}^{1} (q^{s}_{jt+1}+ D_{jt+1})\mathit{v}_{jt} \, dj +(1+i_{t}) \frac{M_{t}}{P_{t+1}}}{A_{t}}$$\\ 

where  $D_{jt+1}$ are dividends of firm $j$ and $i_{t}$ is the nominal interest rate. \\ \\

The mutual fund is risk neutral and looks to maximize its expected return 


$$\max_{\{B_{t}, M_{t} , \mathit{v}_{jt} \}} \mathrm{E}_{t}\left[1+r^{a}_{t+1} \right] = \mathrm{E}\left[ \frac{ B_{t} + \int_{0}^{1} (q^{s}_{jt+1}+ D_{jt+1})\mathit{v}_{jt} \, dj +(1+i_{t}) \frac{M_{t}}{P_{t+1}}}{\frac{M_{t}}{P_{t}} +q^{b}_{t} B_{t} + \int_{0}^{1} q^{s}_{jt}\mathit{v}_{jt}\,dj} \right]$$ \\

 
The first order conditions lead to the no arbitrage equations:

$$ \mathrm{E}_{t}\left[1+r^{a}_{t+1}\right]= \frac{1}{q^{b}_{t}}  =\frac{\mathrm{E}_{t}\left[q^{s}_{jt+1} + D_{jt+1} \right]}{q^{s}_{jt}} = (1+i_{t}) \mathrm{E}_{t}\left[\frac{P_{t}}{P_{t+1}}\right] \equiv 1 +r_{t}$$

where $r_{t}$ is defined to be the real interest rate in period $t$. 

\hypertarget{Firms}{}
\subsection{Firms}

\hypertarget{Final Good Producer}{}
\subsubsection{Final Good Producer}

Perfectly Competitive Final Good Producer aggregates goods with CES technology

$$ Y_{t} = \left(\int_{0}^{1} Y_{jt}^{\frac{\epsilon_{p}-1}{\epsilon_{p}}}\, dj\right)^{\frac{\epsilon_{p}}{\epsilon_{p}-1}}$$

Final Good Producer Profit maximzation Problem

$$ \max_{Y_{jt}} P_{t} \left(\int_{0}^{1} Y_{jt}^{\frac{\epsilon_{p}-1}{\epsilon_{p}}}\, dj\right)^{\frac{\epsilon_{p}}{\epsilon_{p}-1}} - \int_{0}^{1} P_{jt} Y_{jt} ,\ dj $$


This leads to demand

$$ Y_{jt} = \left(\frac {P_{jt}}{P_{t}}\right)^{- \epsilon_{p}} Y_{t}$$

Plugging this demand into $ P_{t}Y_{t} = \int_{0}^{1} P_{jt} Y_{jt} ,\ dj$ , we obtain

Price index $$P_{t} = \left(\int_{0}^{1} P_{jt}^{1-\epsilon_{p}}\,dj \right )^{\frac{1}{1-\epsilon_{p}}}$$


\hypertarget{Intermediate Good Producer}{}
\subsubsection{Intermediate Good Producer}

$$Y_{jt} =  Z_{t}  N_{jt}$$ 

where $log(Z_{t}) = \rho_{Z} log( Z_{t-1}) + \epsilon_{Z}$


 Firm Maximization Problem
 
 Firm $j$ chooses $P_{jt}$ to maximize its dividend $D_{jt}$ and its stock price $q^{s}_{jt} $
 
 $$\max_{\{P_{jt}\}} \overbrace{\frac{(P_{jt} - MC_{t})Y_{jt}}{P_{t}}}^{=D_{jt}} + q^{s}_{jt}\left(P_{jt}\right) $$
 
Given $q^{s}_{jt}\left(P_{jt}\right) = \frac{\mathrm{E}_{t}\left[q^{s}_{jt+1} +D_{jt+1}\left(P_{jt}\right)\right]}{1+r_{t}}$, this is equivalent to: 
 
 $$\max_{\{P_{jt}\}} \mathrm{E}_{t}\left[\sum_{s=0}^{\infty} (\lambda_{P}) ^{s} M_{t,t+s} \left[ \frac{(P_{jt} - MC_{t+s})Y_{jt+s}}{P_{t+s}}\right]\right]$$
 
subject to $$Y_{jt} = \left(\frac {P_{jt}}{P_{t}}\right)^{- \epsilon_{p}} Y_{t}$$
 
where $ \lambda_{P}$ is the  probability a firm cannot change its price,  $M_{t, t+s} = \prod_{k=t}^{t+s-1} \frac{1}{1+r_{k}}$ is the stochastic discount factor and $MC_{t} = \frac{W_{t}}{A_{t}}$ is the marginal cost

Phillips Curve

$$ \pi_{t} = \frac{\mathrm{E}_{t}[\pi_{t+1}]}{1+r^{*}} + \lambda (\mu_{t}-\mu)$$

where $r^{*}$ is the natural rate of interest in the steady state, $\lambda = \frac{(1-\lambda_{p})(1-\frac{\lambda_{p}}{1+r^{*}})}{\lambda_{p}}$,  $ \mu_{t} = log(P_{t}) - log(W_{t}) + log(Z_{t})$ and $\mu = \frac{\epsilon_{p}}{1-\epsilon_{p}}$

\hypertarget{Labor Market}{}
\subsection{Labor Market}


Every worker $i \in [\mho,1]$ provides $n_{igt}$ hours of work to labor union $g \in [0,1]$ and assume $n_{igt} = \mathit{n}_{gt}$. This assumption will imply labor income heterogeneity is only due to transitory income shocks.

Therefore, 

$$n_{it} = \int_{0}^{1} n_{igt}\,dg$$ WLOG assume employed workers are $i \in [\mho,1]$

and 

$$N_{gt} = \int_{\mho}^{1} n_{igt}\,di = (1-\mho_{t}) \mathit{n}_{gt}$$ 

Can think of LHS as labor demand for labor type $g$ o, and RHS is effective labor supply of labor type $g$ from households. There is an underlying assumption that unemployed households supply  "useless" labor. That they're labor supply is effectively useless. Or assumption can be thought as labor union only asks for labor from unemployed households. So if labor unions need to supply $N_{gt}$ to the labor packer, then the labor union will demand $ n_{gt} = \frac{N_{gt}}{1-\mho}$ from each household that is working.


\hypertarget{Competitive Labor Packer}{}
\subsubsection{Competitive Labor Packer}



Perfectly Competitive Labor Packer purchases labor from Labor Unions and  aggregates Labor using CES technology and sells $N_{t}$ to firms at price $W_{t}$


$$ N_{t} = \left(\int_{0}^{1} N_{gt}^{\frac{\epsilon_{w}-1}{\epsilon_{w}}}\,dg\right)^{\frac{\epsilon_{w}}{\epsilon_{w}-1}}$$

The Competitve Labor Packer's profit maximizing Problem 

$$ \max_{n_{jgt}} W_{t} \left(\int_{0}^{1} N_{jgt}^{\frac{\epsilon_{w}-1}{\epsilon_{w}}} \, dg \right)^ {\frac{\epsilon_{w}}{\epsilon_{w}-1}} - \int_{0}^{1} W_{gt}N_{jgt}\, dj $$


 Competitive Labor Packer Demand for labor types

$$ N_{gt} = \left(\frac{W_{gt}}{W_{t}}\right)^{-\epsilon_{w}} N_{t} $$

Wage index follows
$$ W_{t} = \left(\int_{0}^{1} W_{gt}^{1-\epsilon_{w}}\,dg\right)^{\frac{1}{1-\epsilon_{w}}}$$




\hypertarget{Labor Unions}{}
\subsubsection{Labor Unions}

Labor Union Maximization Problem

Labor Union $g$ will set a wage to maximize expected lifetime utility. It may only adjust its price given it is chosen by the calvo fairy. 

$$ \max_{\{W_{gt}\}} \mathrm{E_{t}}\left[\sum_{s=0}^{\infty} (\bar{\beta} \not D \lambda_{w})^{s} \int_{\mho}^{1}  U\left (c_{it+s}(W_{gt+s}), n_{i t+s}) \, di \right)\right] $$

where $\bar{\beta} = \int_{\mho}^{1} \beta_{i} \, di$

subject to the following three constraints $$ N_{gt} = \left(\frac{W_{gt}}{W_{t}}\right)^{-\epsilon_{w}} N_{t} $$

$$ W_{t} = \left(\int_{0}^{1} W_{gt}^{1-\epsilon_{w}}\,dg\right)^{\frac{1}{1-\epsilon_{w}}}$$

where $\lambda_{w}$ probability labor union cannot adjust its wage and $\Phi_{it+s}$ is the distribution of  liquid assets, transitory and permanent shocks over households at period $t+s$. \\


Wage Phillips Curve follows from the first order condition


$$ \pi_{t}^{w} =   \bar{\beta} \not D  \mathrm{E}_{t} \left[ \pi_{t+1}^{w}\right] + \frac{(1-\lambda_{w})}{\lambda_{w}} (1- \bar{\beta} \not D \lambda_{w}) (\mu^{w} - \mu_{t}^{w})$$

where $\mu_{t}^{w} = log\left( \frac{W_{t}}{P_{t}}\right)  - log\left(1 -\tau_{t}\right) - mrs_{t}$


\hypertarget{Government Policy}{}
\subsection{Government Policy}



\hypertarget{Fiscal Policy}{}
\subsubsection{Fiscal Policy}

The government follows the balanced budget

$$ B_{t-1} + G_{t} + \mathit{u} \mho =   q^{b}_{t} B_{t} +  \tau_{t} w_{t} N_{t} $$ 

We will assume $ G_{t} = G$ and $ \tau_{t} = \tau$

\hypertarget{Monetary Policy}{}
\subsubsection{Monetary Policy}


The central bank follows the taylor rule: 

$$i_{t} = r_{t}^{*} +\phi \pi_{t} + \phi_{y} (Y_{t} - Y_{ss}) + v_{t}$$

where $v_{t} = \rho_{v} v_{t-1} +\varepsilon_{t}$


\hypertarget{Equilibrium}{}
\subsection{Equilibrium}


An equilibrium in this economy is a sequence of: \\

- Policy Functions $\left( \mathcal{A}_{t}(m) , \mathcal{C}_{t}(m) \right )_{t=0}^{\infty}$ \\

- Value functions $ \left( V_{t}(m) \right)_{t=0}^{\infty}$\\

- Distributions $ \left(\Phi_{t}(m) \right)_{t=0}^{\infty}$\\

- Prices $ \left( r^{a}_{t}, i_{t}, q^{s}_{t}, q^{b}_{t}, P_{t}, W_{t} , w_{t} , \pi_{t}, \pi^{w}_{t} \right) _{t=0}^{\infty}$\\

- Aggregates $ \left(C_{t}, Y_{t} , N_{t},D_{t} , A_{t} , B_{t} \right)_{t=0}^{\infty}$\\

Such that: \\

$ \left( \mathcal{A}_{t}(m) , \mathcal{C}_{t}(m), V_{t}(m)\right)_{t=0}^{\infty}$  solves the household's maximization problem given $  \left( w_{t}, N_{t},  r^{a}_{t}, \tau_{t}, \Phi_{t-1}(m)\right)_{t=0}^{\infty}$.\\

The Mutual Fund, final Goods producer, intermediate goods producers, labor packer, and labor unions maximize their objective function.

The government budget constraint holds.

The nominal interest rate is set according to the central bank's Taylor rule.


$ \Phi_{t+1}(m) = H(\Phi_{t}(m))$ holds.\\


Markets clear:\\

 $$ A_t =  \int_{0}^{1} \pLevBF_{it}\left( m_{it} - c_{it}(m_{it})\right) \, di $$
 
 $$ Y_t = C_{t} +G $$
 
 where $C_{t} =  \int_{0}^{1} \pLevBF_{it} c_{it}(m_{it})\, di $


\hypertarget{Computation}{}
\section{Computation}

\hypertarget{Calibration}{}
\section{Calibration}

\begin{table}
\begin{center}\renewcommand{\arraystretch}{1.5}
\caption{Household Calibration}\label{table:Calibration}
\begin{tabular}{|c|ccl|c|}
\hline
\multicolumn{5}{|l|}{Calibrated Parameters}  \\ \hline
Description                     & \multicolumn{1}{c}{Parameter} & Value & \multicolumn{2}{c|}{Source/Target }\\ \hline
Coefficient of Relative Risk Aversion & \multicolumn{1}{c}{$\CRRA$} & 2 & \multicolumn{2}{c|}{Conventional} \\
Real Interest Rate                 & \multicolumn{1}{c}{$r$} & $1.048^{.25} - 1$ & \multicolumn{2}{c|}{Conventional} \\
Discount Factor          & \multicolumn{1}{c}{$\beta$} & 0.96 & \multicolumn{2}{c|}{Conventional} \\
Disutility of Labor Coefficient & \multicolumn{1}{c}{$\varphi$} & .883 & \multicolumn{2}{c|}{N = 1.22} \\
Probability of Death       & \multicolumn{1}{c}{$\pZero$} & 0.00625 & \multicolumn{2}{c|}{} \\
Tax Rate       & \multicolumn{1}{c}{$\tau$} & 0.165 & \multicolumn{2}{c|}{} \\
Frisch        & \multicolumn{1}{c}{$\frac{1}{v}$} & .5 & \multicolumn{2}{c|}{Conventional} \\
Unemployment Benefits       & \multicolumn{1}{c}{$u$} & 0.095 & \multicolumn{2}{c|}{} \\
Probability of Unemployment       & \multicolumn{1}{c}{$\mho$} & 0.05 & \multicolumn{2}{c|}{} \\
Std Dev of Log Permanent Shock  & \multicolumn{1}{c}{$\sigma_{\pshk}$} & 0.06 & \multicolumn{2}{c|}{} \\
Std Dev of Log Transitory Shock & \multicolumn{1}{c}{$\sigma_{\theta}$} & 0.2 & \multicolumn{2}{c|}{} \\ \hline
\end{tabular}
\end{center}
\end{table}
\begin{table}
\begin{center}\renewcommand{\arraystretch}{1.5}
\caption{Economy Calibration}\label{table:Calibration}
\begin{tabular}{|c|ccl|c|}
\hline
\multicolumn{5}{|l|}{Calibrated Parameters}  \\ \hline
Description                     & \multicolumn{1}{c}{Parameter} & Value & \multicolumn{2}{c|}{Source/Target }\\ \hline
Calvo Price Stickiness & \multicolumn{1}{c}{$\Lambda_{p}$} & .8 & \multicolumn{2}{c|}{Conventional} \\
Calvo Wage Stickiness                & \multicolumn{1}{c}{$\Lambda_{w}$} & $.7$ & \multicolumn{2}{c|}{} \\
Steady State Price Markup          & \multicolumn{1}{c}{$\mu_{p}$} & 1.012 & \multicolumn{2}{c|}{} \\
Steady State Price Markup          & \multicolumn{1}{c}{$\mu_{w}$} & 1.05 & \multicolumn{2}{c|}{} \\
 Government Spending       & \multicolumn{1}{c}{$G$} & 0.19 & \multicolumn{2}{c|}{} \\
 Steady State Bond Supply       & \multicolumn{1}{c}{$B$} & 0.5 & \multicolumn{2}{c|}{} \\
Taylor Rule Inflation Coefficient        & \multicolumn{1}{c}{$\phi_{\pi}$} & .8 & \multicolumn{2}{c|}{Conventional} \\
 Taylor Rule Output Gap Coefficient       & \multicolumn{1}{c}{$\phi_{y}$} & 0 & \multicolumn{2}{c|}{} \\
Assets to Output Ratio       & \multicolumn{1}{c}{$\frac{A}{Y}$} & 1.4 & \multicolumn{2}{c|}{} \\
Government Bond to Output Ratio & \multicolumn{1}{c}{$\frac{B}{Y}$} & 0.4 & \multicolumn{2}{c|}{} \\ \hline
\end{tabular}
\end{center}
\end{table}






\hypertarget{Results}{}
\section{Results}


\hypertarget{Conclusion}{}
\section{Conclusion}







\providecommand{\figName}{Convergence-of-the-Consumption-Rules}
\providecommand{\figFile}{cFuncsConverge}
\hypertarget{\figFile}{}
\hypertarget{\figName}{}
\begin{figure}[tbp]
\centerline{\includegraphics[width=6in]{\FigDir/\figFile}}
\caption{Convergence of the Consumption Rules}
\label{fig:\figFile}
\end{figure}














\clearpage\vfill\eject

\appendix

\centerline{\LARGE Appendices}\vspace{0.2in}




\clearpage\vfill\eject

\normalsize


\hypertarget{Computational Details}{}
\section{Computational Details}

\hypertarget{Household Bellman Equation }{}
\subsection{Household Bellman Equation}

Household i's dynamic program is

$$ V(\pmb{\mathrm{m}}_{it},\pmb{\mathrm{p}}_{it})=\max_{\{ \pmb{\mathrm{c}}_{it}\}} { U( \pmb{\mathrm{c}}_{it}, n_{it}) + \beta_{i} \not D \mathrm{E}_{t}[V( \pmb{\mathrm{m}}_{it+1} , \pmb{\mathrm{p}}_{it+1})]}$$

subject to 

\begin{align*}
 \pmb{\mathrm{m}}_{i t} & = \pmb{\mathrm{z}}_{i t}  + (1+\mathit{r}^{a}_{t})\pmb{\mathrm{a}}_{i t-1} \\
 \pmb{\mathrm{c}}_{i t}  + \pmb{\mathrm{a}}_{i t} &= \pmb{\mathrm{z}}_{i t}  + (1+\mathit{r}^{a}_{t}) \pmb{\mathrm{a}}_{i t-1}   \\
\pmb{\mathrm{a}}_{it} &\geq 0 
\end{align*} \\ \\

This can be normalized to \\


$$ V(m_{it}) = \max_{\{c_{it}\}} { U(c_{it}, n_{it}) + \beta_{i}\not D \mathrm{E}_{t}[\psi_{it+1}^{1-\rho} V(m_{it+1})]}$$

 subject to 
 
 \begin{align*}
m_{i t} &=  \xi_{it}  + (1+r^{a}_{t}) \frac{a_{i t-1}}{\psi_{it-1}} \\
 c_{i t}  + a_{i t} &= \xi_{it}  + (1+r^{a}_{t}) \frac{a_{i t-1}}{\psi_{it-1}} \\
 a_{it} &\geq 0 
 \end{align*}
 
 Here non boldface variables are normalized by permanent income $\mathit{p_{it}}$. 

e.g. $x_{it} = \frac{\mathbf{x_{it}}}{\pmb{\mathrm{p}}_{it}}$




\hypertarget{Model as System}{}
\subsection{Model as System}

$$
H_{t}(\mathbf{U},\mathbf{Z})= \begin{pmatrix} 
 Y_{t} - Z_{t}N_{t} \\ \\ 
B_{t-1} - q^{b}_{t}B_{t} + u\mho + G_{t} - \tau w_{t} N_{t} \\ \\  
i_{t} - r^{*} - \phi \pi_{t} -\phi_{y}(Y_{t}-Y_{ss}) - v_{t} \\ \\
\pi_{t} -\frac{\pi_{t+1}}{1+r^{*}} + \lambda(\mu_{t}^{p} -\mu_{p})  \\ \\
 \pi_{t}^{w} -\not D \pi_{t+1}^{w} -(\frac{1-\lambda_{w}}{\lambda_{w}}) (1- \not D \lambda_{w}) (\mu^{w} -\mu_{t}^{w}) \\ \\
    1+r_{t} - \frac{1 + i_{t}}{1+ \pi^{p}_{t+1}}\\ \\
 1+r_{t+1}^{a} - \frac{q_{t+1}^{s} +D_{t+1}}{q_{t}^{s}} \\ \\
 r_{t} - r_{t+1}^{a} \\ \\
 \frac{w_{t}}{w_{t-1}} - \frac{\Pi_{t}^{w}}{\Pi_{t}^{p}} \\ \\
 \mathcal{C}_{t}(\{r_{s}^{a} ,w_{s}, N_{s}\}_{s=0}^{s=T}) - Y_{t} + G_{t}  \\ \\
  \mathcal{A}(\{r_{s}^{a} ,w_{s}, N_{s}\}_{s=0}^{s=T}) - q_{t}^{s} - \frac{B_{t}}{1+r_{t}}  \\ \\
 \end{pmatrix} = \begin{pmatrix} 0 \\ 0 \\. \\. \\. \\ 0\\ \end{pmatrix} , \quad t=0,1 ,2,3,....
$$ \\ \\
 

 
 where \\
 
 $\mathcal{C}_{t}(\{r_{s}^{a} ,w_{s}, N_{s}\}_{s=0}^{s=T}) = \int_{0}^{1} \pLevBF_{it} c_{it}(m_{it})\, di $ \\
 
 $\mathcal{A}_{t}(\{r_{s}^{a} ,w_{s}, N_{s}\}_{s=0}^{s=T}) = \int_{0}^{1} \pLevBF_{it} \left(m_{it} - c_{it}(m_{it})\right)\, di $ \\
 
$c_{it}(m_{it})$ is the steady state normalized consumption policy for household $i$ in period t. \\ \\
 

 
 
 $\mathbf{U} = \left(Y_{t} , N_{t} ,  D_{t
 }, B_{t}, w_{t} , \pi_{t}^{p} ,\pi_{t}^{w}, r_{t} , r_{t+1}^{a}, i_{t} , q_{t}^{s},  q_{t}^{s} \right)_{t=0}^{t=T}$ \\ 

 
 $\mathbf{Z} = \left(Z_{t} ,v_{t}\right)_{t=0}^{t=T}$ \\ \\
 
 
\hypertarget{Reduced System}{}
\subsubsection{Reduced System}
 
 System Can be reduced to the following: \\ \\
 
Exogenous Variables are $ Z_{t}, v_{t}$ \\ 

Endogenous Variables are $ r_{t} , w_{t} ,N_{t}$ \\ \\

\begin{eqnarray} 
H_{t}(\mathbf{U},\mathbf{Z})= \begin{pmatrix} 
\mathcal{H}_{t,1} \\ \\ 
\mathcal{H}_{t,2} \\ \\
\mathcal{H}_{t,3} \\ \\
 \end{pmatrix} = \begin{pmatrix} 0 \\ 0 \\ 0 \\ \end{pmatrix} , \quad  t = 0, 1, 2, ..., T 
 \end{eqnarray}
 
 where \\ 
 
 
 $\mathcal{H}_{t,1}  =\mathcal{C}_{t}\left( \left \{r^{a}_{s} , w_{s} , N_{s}  \right \}_{s=0}^{s=T} \right) - Z_{t} N_{t} + G\\ \\ $

$ \mathcal{H}_{t,2}  =log(w_{t}) - log(w_{t-1}) + \left( \frac{1 - \lambda_{w}}{\lambda_{w}}(1 - \not D \lambda_{w}) \sum_{k=0}^{\infty} \not D^{k} ( \mu_{t+k}^{w} - \mu^{w}) \right) - \left(  \lambda \sum_{k=0}^{\infty} \frac{1}{(1+r^{*})^{k}} ( \mu_{t+k}^{p} - \mu^{p})\right)\\ \\ $

$ \mathcal{H}_{t,3}  =  (1+r_{t}) \left(1+ -\lambda \sum_{k=1}^{\infty} \frac{1}{(1+r^{*})^{k}} ( \mu_{t+k}^{p} - \mu^{p}) \right) \\ \\
 - \left(1+r^{*}+ \phi \left(- \lambda \sum_{k=0}^{\infty} \frac{1}{(1+r^{*})^{k}} ( \mu_{t+k}^{p} - \mu^{p}) \right) +\phi_{y} \left(Z_{t} N_{t} - Y_{ss} \right) + v_{t}\right) \\ \\ $
 
$\mathbf{U} = (r_{0} , r_{1} , ...r_{T}, w_{0}, w_{1}, ..., w_{T}, N_{0}, N_{1},...,N_{T})$ \\ 

$ \mathbf{Z} = ( Z_{0}, Z_{1},... Z_{T}, v_{0},...,v_{T}) \\ \\ $ 

$ \mu_{t}^{p} = log(\frac{1}{w_{t}}) + log(Z_{t})$ \\ 

$\mu_{t}^{w} = log(w_{t}) + log(1 - \tau_{t}) - mrs_{t} \\ \\ $


$mrs_{t} = log \left(- \frac{\int_{0}^{1}   U_{n} \left(\cLevBF_{i t}, n_{i t} \right) \ d i  }{\int_{0}^{1} \pLevBF_{it} \theta_{it} U_{c} \left(\cLevBF_{i t}, n_{i t} \right) \  di } \right) = log \left(\frac{\int_{0}^{1} \varphi \pLevBF_{it} n_{it}^{v} \ d i  }{\int_{0}^{1} \pLevBF_{it}  \theta_{it} \cLevBF_{it}^{-\rho} \  di } \right) = log \left(\frac{\int_{0}^{1} \varphi \pLevBF_{it} \left(\frac{N_{t}}{1 - \mho} \right)^{v} \, d i  }{\int_{0}^{1} \pLevBF_{it}  \theta_{it} \cLevBF_{it}^{-\rho} \  di } \right)$ \\ 

$ = log \left( \varphi \left(\frac{N_{t}}{1 - \mho}\right) ^{v}\right) + log \left(\int_{0}^{1} \pLevBF_{it}  \,  di  \right) - log \left(\int_{0}^{1} \pLevBF_{it}  \theta_{it} \cLevBF_{it}^{-\rho} \  di  \right) $


\hypertarget{Jacobian of System}{}
\subsection{Jacobian of System} 

By applying implicit function theorem to equation 10 : \\ \\

$$d\mathbf{U} =  -{\mathbf{H}_{\mathbf{U}}}^{-1} \mathbf{H}_{\mathbf{Z}} d \mathbf{Z}$$ \\ 


 $$  \mathbf{H}_{\mathbf{U}}= \begin{pmatrix} 
H_{\mathbf{u}, 0} \\ \\ 
H_{\mathbf{u}, 1}  \\ \\
. \\ \\
. \\ \\
. \\ \\ 
H_{\mathbf{u}, T} \\ \\
 \end{pmatrix} \quad \quad \mathbf{H}_{\mathbf{Z}}= \begin{pmatrix} 
H_{\mathbf{Z}, 0} \\ \\ 
H_{\mathbf{Z}, 1}  \\ \\
. \\ \\
. \\ \\
. \\ \\ 
H_{\mathbf{Z}, T} \\ \\
 \end{pmatrix}$$ \\ \\
 
 
 
$$ H_{\mathbf{U}, t}= \begin{pmatrix} 
\frac{ \partial \mathcal{H}_{t,1}}{\partial r_{0}}  & ... & \frac{ \partial \mathcal{H}_{t,1}}{\partial r_{T}} & \frac{ \partial \mathcal{H}_{t,1}}{\partial w_{0}} & ... & \frac{ \partial \mathcal{H}_{t,1}}{\partial w_{T}} & \frac{ \partial \mathcal{H}_{t,1}}{\partial N_{0}} & ... &\frac{ \partial \mathcal{H}_{t,1}}{\partial N_{T}} \\ \\ 
\frac{ \partial \mathcal{H}_{t,2}}{\partial r_{0}}  & ... & \frac{ \partial \mathcal{H}_{t,2}}{\partial r_{T}} & \frac{ \partial \mathcal{H}_{t,2}}{\partial w_{0}} & ... & \frac{ \partial \mathcal{H}_{t,2}}{\partial w_{T}} & \frac{ \partial \mathcal{H}_{t,2}}{\partial N_{0}} & ... &\frac{ \partial \mathcal{H}_{t,2}}{\partial N_{T}}  \\ \\
\frac{ \partial \mathcal{H}_{t,3}}{\partial r_{0}}  & ... & \frac{ \partial \mathcal{H}_{t,3}}{\partial r_{T}} & \frac{ \partial \mathcal{H}_{t,3}}{\partial w_{0}} & ... & \frac{ \partial \mathcal{H}_{t,3}}{\partial w_{T}} & \frac{ \partial \mathcal{H}_{t,3}}{\partial N_{0}} & ... &\frac{ \partial \mathcal{H}_{t,3}}{\partial N_{T}} \\ \\
 \end{pmatrix} $$ \\
 
  $$ H_{\mathbf{Z}, t}= \begin{pmatrix} 
\frac{ \partial \mathcal{H}_{t,1}}{\partial Z_{0}}  & ... & \frac{ \partial \mathcal{H}_{t,1}}{\partial Z_{T}} & \frac{ \partial \mathcal{H}_{t,1}}{\partial v_{0}} & ... & \frac{ \partial \mathcal{H}_{t,1}}{\partial v_{T}} \\ \\ 
\frac{ \partial \mathcal{H}_{t,2}}{\partial Z_{0}}  & ... & \frac{ \partial \mathcal{H}_{t,2}}{\partial Z_{T}} & \frac{ \partial \mathcal{H}_{t,2}}{\partial v_{0}} & ... & \frac{ \partial \mathcal{H}_{t,2}}{\partial v_{T}} \\ \\
\frac{ \partial \mathcal{H}_{t,3}}{\partial Z_{0}}  & ... & \frac{ \partial \mathcal{H}_{t,3}}{\partial Z_{T}} & \frac{ \partial \mathcal{H}_{t,3}}{\partial v_{0}} & ... & \frac{ \partial \mathcal{H}_{t,3}}{\partial v_{T}}  \\ \\
 \end{pmatrix} $$
 
 

\bibliography{\econtexRoot/BufferStockTheory,economics}

\end{document}


\provideboolean{Shorter}
\setboolean{Shorter}{true}
\setboolean{Shorter}{false}
\providecommand{\ShorterYN}{\ifthenelse{\boolean{Shorter}}}
\usepackage{rotating}\usepackage{subfigure}


\hypersetup{pdfauthor={William Du <wdu9@jhu.edu>},
            pdftitle={Theoretical Foundations of Buffer Stock Saving},
            pdfkeywords={Precautionary saving, buffer-stock saving, consumption, marginal propensity to consume, permanent income hypothesis},
            pdfcreator = {wdu9@jhu.edu}
}

\begin{document}\bibliographystyle{\econtexBibStyle}
\renewcommand{\onlyinsubfile}[1]{}\renewcommand{\notinsubfile}[1]{#1} 

\hfill{\tiny \texname.tex, \today}

\begin{verbatimwrite}{\texname.title}
Theoretical Foundations of Buffer Stock Saving
\end{verbatimwrite}


\title{Distribution of Wealth and Monetary Policy}

\author{William Du\authNum}

\keywords{Precautionary saving, Heterogeneous Agents, Monetary Policy, permanent income hypothesis}

\jelclass{D81, D91, E21}


\maketitle 


\hypertarget{abstract}{}
\begin{abstract}
  This paper develops a heterogenous Agent New Keynesian Model with a friedman buffer stock income process.
\end{abstract}

\begin{small}
\parbox{\textwidth}{
\begin{center}
\begin{tabbing}
\texttt{~Archive:~} \= \= \url{} \kill \\  %
\texttt{~~~~~PDF:~} \> \> \url{} \\
\texttt{~~Slides:~} \> \> \url{} \\
\texttt{~~~~~Web:~} \> \> \url{}    \\
\texttt{~~GitHub:~} \> \> \url{http://github.com/wdu9/FBS-NK} \\
\texttt{~~~~~~~~~~} \> \> \textit{(In GitHub repo, see \texttt{/Code} for tools for solving and simulating the model)} \\
\end{tabbing}
\end{center}
          
\href{https://colab.research.google.com/github/econ-ark/REMARK/blob/master/REMARKs/BufferStockTheory/BufferStockTheory.ipynb}{CLICK HERE} for an interactive \href{http:https://en.wikipedia.org/wiki/Project_Jupyter}{Jupyter Notebook} that uses the \href{https://econ-ark/HARK}{Econ-ARK/HARK} toolkit (\cite{carroll_et_al-proc-scipy-2018}) to produce all of the paper's figures (warning: it may take several minutes to launch)
}
\end{small}

\begin{authorsinfo}
\name{Contact: \href{mailto:wdu9@jhu.edu}{\texttt{wdu9@jhu.edu}}}
\end{authorsinfo}

\thanks{Thanks to }

\titlepagefinish


\newtheorem{defn}{Definition}
\newtheorem{theorem}{Theorem}

\hypertarget{Introduction}{}
\section{Introduction}

\label{sec:intro}


Write here for intro



\hypertarget{The-Model}{}
\section{The Model}

\subsection{Households}
\label{subsec:Households} 

There is a continuum of households of mass 1 distributed on the unit
interval and indexed by $i$. Households are ex-ante heterogeneous in their discount factors and are subject to idiosyncratic income shocks.  Each household faces the following problem:

\begin{verbatimwrite}{\EqDir/supfn.tex}
\begin{eqnarray}
  \label{eq:supfn}
  \max_{\{\cLevBF_{it+s}\}_{s=0}^{\infty}} \mathrm{E_{t}}\left[\sum_{s=0}^{\infty} (\not D \beta_{i})^{t+s} U\left(  \cLevBF_{i t+s}, n_{i t+s}\right)\right]
\end{eqnarray}
\end{verbatimwrite}
\input{\EqDir/supfn.tex} 

subject to 
\begin{align*}
\aLevBF_{it}     &= \mLevBF_{it} - \cLevBF_{it}   \label{eq:DBCparts} \\
\aLevBF_{it} +\cLevBF_{it}    &= \mathbf{z}_{it} +   (1 + r^{a}_{t} ) \aLevBF_{it-1}   \\ 
\aLevBF_{it}  &\geq 0 \\
\end{align*}

where
$U\left(\cLevBF_{i t}, n_{i t}\right) = \frac{\cLevBF_{i t}^{1-\rho}}{1 -\rho} - \varphi \pLevBF_{it} \frac{n_{it}^{1+v}}{1+v}$ , $\beta_{i}$ is the discount factor of household $i$ and $\not D$ is the probability of death.  \\

$\mLevBF_{it}$ \ denotes household $i$'s market resources at time $t$ to be expended on consumption $\cLevBF_{it}$ or invested into an asset $ \aLevBF_{it}$ with return $r_{t+1}^{a}$.  $\mLevBF_{it}$ is determined by labor income,  $\mathbf{z}_{it}$, and the gross return on assets from the last period, $(1+r_{t}^{a}) \aLevBF_{it-1} $. Labor supply of household $i$ at time $t$ is denoted by $n_{it}$.  Given the formulation of sticky wages described in section 2.4, labor supply is an aggregate state variable and therefore consumption serves as the sole control variable in the dynamic problem. \\




\begin{align*}
\mathbf{z}_{it} &= \pLevBF_{it}\tShkAll_{it} \\
\pLevBF_{it+1} &=\pLevBF_{it} \pShk_{it+1} \\
\end{align*}


Labor income is subject to permanent and transitory idiosyncratic shocks. In particular, household $i$'s labor income is composed of a permanent component, $\pLevBF_{it} $ indicating the level of permanent income and a transitory component, $\tShkAll_{it} $, indicating the transitory income shock received by household $i$ at time $t$. $\pLevBF_{it} $ is subject to permanent income shocks $\pShk_{it+1}$ where $\pShk_{it}$ is iid mean one lognormal with standard deviation $\sigma_\pShk$, $\forall t$  
($\Ex_{t}[{\pShk}_{t+n}]=1~\forall~n>0$) .



The transitory random variable follows   
\begin{verbatimwrite}{\EqDir/tShkDef}
\begin{equation}
\tShkAll _{it+n}=
\begin{cases}
 u \phantom{_{t+1}/\pNotZero} & \text{with probability $\pZero>0$} \\
 \tShkEmp_{it+n} (1-\tau_{t})\int_{0}^{1} w_{gt}n_{igt} \, dg      & \text{with probability $\pNotZero  $} 
\end{cases} \label{eq:tShkDef}
\end{equation}
\end{verbatimwrite}
\input{\EqDir/tShkDef.tex}
where $\tau_{t}$ is the tax rate , $w_{gt}$ is the real wage for labor type $g$ at time t, $ n_{igt}$ is the labor supply for labor type $g$ and $\tShkEmp_{t+n}$ is an iid mean-one lognormal with standard deviation $\sigma_{\tShkEmp}$,
($\Ex_{t}[{\tShkEmp}_{t+n}]=1~\forall~n>0$).




\begin{comment}
Combining the transition equations, the recursive nature of
the problem allows us to rewrite it more compactly in Bellman equation form,
\begin{eqnarray*}
\VFunc_{t}(\mLevBF_{t},\pLevBF_{t}) & = & \max_{\cLevBF_{t}}~\left\{\util(\cLevBF_{t})+\DiscFac \Ex_{t}\left[ \VFunc_{t+1}((\mLevBF_{t}-\cLevBF_{t})\Rfree+ \pLevBF_{t+1}\tShkAll_{t+1},\pLevBF_{t} \PGro  \pShk_{t+1})\right]\right\}
.
\end{eqnarray*}
\end{comment}

\hypertarget{Financial Intermediary}{}
\subsection{Financial Intermediary}

\label{subsec:Financial Intermediary}

The financial intermediary in our model performs a mutual fund activity where it  collects assets from households $A_{t}$ and invests them into government bonds $B_{t}$, stocks $v_{jt}$, and nominal reserves at the central bank $M_{t}$.\\ 

In particular, at the end of period $t$, the assets collected from households $A_{t}$ must be invested into shares $\mathit{v}_{jt}$ of firm $j$ at price  $q^{s}_{jt}$ , government bonds $B_{t}$ at price $q^{b}_{t}$ and nominal reserves $M_{t}$. 

$$A_{t} = \frac{M_{t}}{P_{t}} +q^{b}_{t} B_{t} + \int_{0}^{1} q^{s}_{jt}\mathit{v}_{jt}\,dj$$

where $A_{t} = \int_{0}^{1} a_{it} \, di$ \\

The mutual fund's return in the next period is then 

$$(1+r^{a}_{t+1})  = \frac{  B_{t} + \int_{0}^{1} (q^{s}_{jt+1}+ D_{jt+1})\mathit{v}_{jt} \, dj +(1+i_{t}) \frac{M_{t}}{P_{t+1}}}{A_{t}}$$\\ 

where  $D_{jt+1}$ are dividends of firm $j$ and $i_{t}$ is the nominal interest rate. \\ \\

The mutual fund is risk neutral and looks to maximize its expected return 


$$\max_{\{B_{t}, M_{t} , \mathit{v}_{jt} \}} \mathrm{E}_{t}\left[1+r^{a}_{t+1} \right] = \mathrm{E}\left[ \frac{ B_{t} + \int_{0}^{1} (q^{s}_{jt+1}+ D_{jt+1})\mathit{v}_{jt} \, dj +(1+i_{t}) \frac{M_{t}}{P_{t+1}}}{\frac{M_{t}}{P_{t}} +q^{b}_{t} B_{t} + \int_{0}^{1} q^{s}_{jt}\mathit{v}_{jt}\,dj} \right]$$ \\

 
The first order conditions lead to the no arbitrage equations:

$$ \mathrm{E}_{t}\left[1+r^{a}_{t+1}\right]= \frac{1}{q^{b}_{t}}  =\frac{\mathrm{E}_{t}\left[q^{s}_{jt+1} + D_{jt+1} \right]}{q^{s}_{jt}} = (1+i_{t}) \mathrm{E}_{t}\left[\frac{P_{t}}{P_{t+1}}\right] \equiv 1 +r_{t}$$

where $r_{t}$ is defined to be the real interest rate in period $t$. 

\hypertarget{Firms}{}
\subsection{Firms}

\hypertarget{Final Good Producer}{}
\subsubsection{Final Good Producer}

Perfectly Competitive Final Good Producer aggregates goods with CES technology

$$ Y_{t} = \left(\int_{0}^{1} Y_{jt}^{\frac{\epsilon_{p}-1}{\epsilon_{p}}}\, dj\right)^{\frac{\epsilon_{p}}{\epsilon_{p}-1}}$$

Final Good Producer Profit maximzation Problem

$$ \max_{Y_{jt}} P_{t} \left(\int_{0}^{1} Y_{jt}^{\frac{\epsilon_{p}-1}{\epsilon_{p}}}\, dj\right)^{\frac{\epsilon_{p}}{\epsilon_{p}-1}} - \int_{0}^{1} P_{jt} Y_{jt} ,\ dj $$


This leads to demand

$$ Y_{jt} = \left(\frac {P_{jt}}{P_{t}}\right)^{- \epsilon_{p}} Y_{t}$$

Plugging this demand into $ P_{t}Y_{t} = \int_{0}^{1} P_{jt} Y_{jt} ,\ dj$ , we obtain

Price index $$P_{t} = \left(\int_{0}^{1} P_{jt}^{1-\epsilon_{p}}\,dj \right )^{\frac{1}{1-\epsilon_{p}}}$$


\hypertarget{Intermediate Good Producer}{}
\subsubsection{Intermediate Good Producer}

$$Y_{jt} =  Z_{t}  N_{jt}$$ 

where $log(Z_{t}) = \rho_{Z} log( Z_{t-1}) + \epsilon_{Z}$


 Firm Maximization Problem
 
 Firm $j$ chooses $P_{jt}$ to maximize its dividend $D_{jt}$ and its stock price $q^{s}_{jt} $
 
 $$\max_{\{P_{jt}\}} \overbrace{\frac{(P_{jt} - MC_{t})Y_{jt}}{P_{t}}}^{=D_{jt}} + q^{s}_{jt}\left(P_{jt}\right) $$
 
Given $q^{s}_{jt}\left(P_{jt}\right) = \frac{\mathrm{E}_{t}\left[q^{s}_{jt+1} +D_{jt+1}\left(P_{jt}\right)\right]}{1+r_{t}}$, this is equivalent to: 
 
 $$\max_{\{P_{jt}\}} \mathrm{E}_{t}\left[\sum_{s=0}^{\infty} (\lambda_{P}) ^{s} M_{t,t+s} \left[ \frac{(P_{jt} - MC_{t+s})Y_{jt+s}}{P_{t+s}}\right]\right]$$
 
subject to $$Y_{jt} = \left(\frac {P_{jt}}{P_{t}}\right)^{- \epsilon_{p}} Y_{t}$$
 
where $ \lambda_{P}$ is the  probability a firm cannot change its price,  $M_{t, t+s} = \prod_{k=t}^{t+s-1} \frac{1}{1+r_{k}}$ is the stochastic discount factor and $MC_{t} = \frac{W_{t}}{A_{t}}$ is the marginal cost

Phillips Curve

$$ \pi_{t} = \frac{\mathrm{E}_{t}[\pi_{t+1}]}{1+r^{*}} + \lambda (\mu_{t}-\mu)$$

where $r^{*}$ is the natural rate of interest in the steady state, $\lambda = \frac{(1-\lambda_{p})(1-\frac{\lambda_{p}}{1+r^{*}})}{\lambda_{p}}$,  $ \mu_{t} = log(P_{t}) - log(W_{t}) + log(Z_{t})$ and $\mu = \frac{\epsilon_{p}}{1-\epsilon_{p}}$

\hypertarget{Labor Market}{}
\subsection{Labor Market}


Every worker $i \in [\mho,1]$ provides $n_{igt}$ hours of work to labor union $g \in [0,1]$ and assume $n_{igt} = \mathit{n}_{gt}$. This assumption will imply labor income heterogeneity is only due to transitory income shocks.

Therefore, 

$$n_{it} = \int_{0}^{1} n_{igt}\,dg$$ WLOG assume employed workers are $i \in [\mho,1]$

and 

$$N_{gt} = \int_{\mho}^{1} n_{igt}\,di = (1-\mho_{t}) \mathit{n}_{gt}$$ 

Can think of LHS as labor demand for labor type $g$ o, and RHS is effective labor supply of labor type $g$ from households. There is an underlying assumption that unemployed households supply  "useless" labor. That they're labor supply is effectively useless. Or assumption can be thought as labor union only asks for labor from unemployed households. So if labor unions need to supply $N_{gt}$ to the labor packer, then the labor union will demand $ n_{gt} = \frac{N_{gt}}{1-\mho}$ from each household that is working.


\hypertarget{Competitive Labor Packer}{}
\subsubsection{Competitive Labor Packer}



Perfectly Competitive Labor Packer purchases labor from Labor Unions and  aggregates Labor using CES technology and sells $N_{t}$ to firms at price $W_{t}$


$$ N_{t} = \left(\int_{0}^{1} N_{gt}^{\frac{\epsilon_{w}-1}{\epsilon_{w}}}\,dg\right)^{\frac{\epsilon_{w}}{\epsilon_{w}-1}}$$

The Competitve Labor Packer's profit maximizing Problem 

$$ \max_{n_{jgt}} W_{t} \left(\int_{0}^{1} N_{jgt}^{\frac{\epsilon_{w}-1}{\epsilon_{w}}} \, dg \right)^ {\frac{\epsilon_{w}}{\epsilon_{w}-1}} - \int_{0}^{1} W_{gt}N_{jgt}\, dj $$


 Competitive Labor Packer Demand for labor types

$$ N_{gt} = \left(\frac{W_{gt}}{W_{t}}\right)^{-\epsilon_{w}} N_{t} $$

Wage index follows
$$ W_{t} = \left(\int_{0}^{1} W_{gt}^{1-\epsilon_{w}}\,dg\right)^{\frac{1}{1-\epsilon_{w}}}$$




\hypertarget{Labor Unions}{}
\subsubsection{Labor Unions}

Labor Union Maximization Problem

Labor Union $g$ will set a wage to maximize expected lifetime utility. It may only adjust its price given it is chosen by the calvo fairy. 

$$ \max_{\{W_{gt}\}} \mathrm{E_{t}}\left[\sum_{s=0}^{\infty} (\bar{\beta} \not D \lambda_{w})^{s} \int_{\mho}^{1}  U\left (c_{it+s}(W_{gt+s}), n_{i t+s}) \, di \right)\right] $$

where $\bar{\beta} = \int_{\mho}^{1} \beta_{i} \, di$

subject to the following three constraints $$ N_{gt} = \left(\frac{W_{gt}}{W_{t}}\right)^{-\epsilon_{w}} N_{t} $$

$$ W_{t} = \left(\int_{0}^{1} W_{gt}^{1-\epsilon_{w}}\,dg\right)^{\frac{1}{1-\epsilon_{w}}}$$

where $\lambda_{w}$ probability labor union cannot adjust its wage and $\Phi_{it+s}$ is the distribution of  liquid assets, transitory and permanent shocks over households at period $t+s$. \\


Wage Phillips Curve follows from the first order condition


$$ \pi_{t}^{w} =   \bar{\beta} \not D  \mathrm{E}_{t} \left[ \pi_{t+1}^{w}\right] + \frac{(1-\lambda_{w})}{\lambda_{w}} (1- \bar{\beta} \not D \lambda_{w}) (\mu^{w} - \mu_{t}^{w})$$

where $\mu_{t}^{w} = log\left( \frac{W_{t}}{P_{t}}\right)  - log\left(1 -\tau_{t}\right) - mrs_{t}$


\hypertarget{Government Policy}{}
\subsection{Government Policy}



\hypertarget{Fiscal Policy}{}
\subsubsection{Fiscal Policy}

The government follows the balanced budget

$$ B_{t-1} + G_{t} + \mathit{u} \mho =   q^{b}_{t} B_{t} +  \tau_{t} w_{t} N_{t} $$ 

We will assume $ G_{t} = G$ and $ \tau_{t} = \tau$

\hypertarget{Monetary Policy}{}
\subsubsection{Monetary Policy}


The central bank follows the taylor rule: 

$$i_{t} = r_{t}^{*} +\phi \pi_{t} + \phi_{y} (Y_{t} - Y_{ss}) + v_{t}$$

where $v_{t} = \rho_{v} v_{t-1} +\varepsilon_{t}$


\hypertarget{Equilibrium}{}
\subsection{Equilibrium}


An equilibrium in this economy is a sequence of: \\

- Policy Functions $\left( \mathcal{A}_{t}(m) , \mathcal{C}_{t}(m) \right )_{t=0}^{\infty}$ \\

- Value functions $ \left( V_{t}(m) \right)_{t=0}^{\infty}$\\

- Distributions $ \left(\Phi_{t}(m) \right)_{t=0}^{\infty}$\\

- Prices $ \left( r^{a}_{t}, i_{t}, q^{s}_{t}, q^{b}_{t}, P_{t}, W_{t} , w_{t} , \pi_{t}, \pi^{w}_{t} \right) _{t=0}^{\infty}$\\

- Aggregates $ \left(C_{t}, Y_{t} , N_{t},D_{t} , A_{t} , B_{t} \right)_{t=0}^{\infty}$\\

Such that: \\

$ \left( \mathcal{A}_{t}(m) , \mathcal{C}_{t}(m), V_{t}(m)\right)_{t=0}^{\infty}$  solves the household's maximization problem given $  \left( w_{t}, N_{t},  r^{a}_{t}, \tau_{t}, \Phi_{t-1}(m)\right)_{t=0}^{\infty}$.\\

The Mutual Fund, final Goods producer, intermediate goods producers, labor packer, and labor unions maximize their objective function.

The government budget constraint holds.

The nominal interest rate is set according to the central bank's Taylor rule.


$ \Phi_{t+1}(m) = H(\Phi_{t}(m))$ holds.\\


Markets clear:\\

 $$ A_t =  \int_{0}^{1} \pLevBF_{it}\left( m_{it} - c_{it}(m_{it})\right) \, di $$
 
 $$ Y_t = C_{t} +G $$
 
 where $C_{t} =  \int_{0}^{1} \pLevBF_{it} c_{it}(m_{it})\, di $


\hypertarget{Computation}{}
\section{Computation}

\hypertarget{Calibration}{}
\section{Calibration}

\begin{table}
\begin{center}\renewcommand{\arraystretch}{1.5}
\caption{Household Calibration}\label{table:Calibration}
\begin{tabular}{|c|ccl|c|}
\hline
\multicolumn{5}{|l|}{Calibrated Parameters}  \\ \hline
Description                     & \multicolumn{1}{c}{Parameter} & Value & \multicolumn{2}{c|}{Source/Target }\\ \hline
Coefficient of Relative Risk Aversion & \multicolumn{1}{c}{$\CRRA$} & 2 & \multicolumn{2}{c|}{Conventional} \\
Real Interest Rate                 & \multicolumn{1}{c}{$r$} & $1.048^{.25} - 1$ & \multicolumn{2}{c|}{Conventional} \\
Discount Factor          & \multicolumn{1}{c}{$\beta$} & 0.96 & \multicolumn{2}{c|}{Conventional} \\
Disutility of Labor Coefficient & \multicolumn{1}{c}{$\varphi$} & .883 & \multicolumn{2}{c|}{N = 1.22} \\
Probability of Death       & \multicolumn{1}{c}{$\pZero$} & 0.00625 & \multicolumn{2}{c|}{} \\
Tax Rate       & \multicolumn{1}{c}{$\tau$} & 0.165 & \multicolumn{2}{c|}{} \\
Frisch        & \multicolumn{1}{c}{$\frac{1}{v}$} & .5 & \multicolumn{2}{c|}{Conventional} \\
Unemployment Benefits       & \multicolumn{1}{c}{$u$} & 0.095 & \multicolumn{2}{c|}{} \\
Probability of Unemployment       & \multicolumn{1}{c}{$\mho$} & 0.05 & \multicolumn{2}{c|}{} \\
Std Dev of Log Permanent Shock  & \multicolumn{1}{c}{$\sigma_{\pshk}$} & 0.06 & \multicolumn{2}{c|}{} \\
Std Dev of Log Transitory Shock & \multicolumn{1}{c}{$\sigma_{\theta}$} & 0.2 & \multicolumn{2}{c|}{} \\ \hline
\end{tabular}
\end{center}
\end{table}
\begin{table}
\begin{center}\renewcommand{\arraystretch}{1.5}
\caption{Economy Calibration}\label{table:Calibration}
\begin{tabular}{|c|ccl|c|}
\hline
\multicolumn{5}{|l|}{Calibrated Parameters}  \\ \hline
Description                     & \multicolumn{1}{c}{Parameter} & Value & \multicolumn{2}{c|}{Source/Target }\\ \hline
Calvo Price Stickiness & \multicolumn{1}{c}{$\Lambda_{p}$} & .8 & \multicolumn{2}{c|}{Conventional} \\
Calvo Wage Stickiness                & \multicolumn{1}{c}{$\Lambda_{w}$} & $.7$ & \multicolumn{2}{c|}{} \\
Steady State Price Markup          & \multicolumn{1}{c}{$\mu_{p}$} & 1.012 & \multicolumn{2}{c|}{} \\
Steady State Price Markup          & \multicolumn{1}{c}{$\mu_{w}$} & 1.05 & \multicolumn{2}{c|}{} \\
 Government Spending       & \multicolumn{1}{c}{$G$} & 0.19 & \multicolumn{2}{c|}{} \\
 Steady State Bond Supply       & \multicolumn{1}{c}{$B$} & 0.5 & \multicolumn{2}{c|}{} \\
Taylor Rule Inflation Coefficient        & \multicolumn{1}{c}{$\phi_{\pi}$} & .8 & \multicolumn{2}{c|}{Conventional} \\
 Taylor Rule Output Gap Coefficient       & \multicolumn{1}{c}{$\phi_{y}$} & 0 & \multicolumn{2}{c|}{} \\
Assets to Output Ratio       & \multicolumn{1}{c}{$\frac{A}{Y}$} & 1.4 & \multicolumn{2}{c|}{} \\
Government Bond to Output Ratio & \multicolumn{1}{c}{$\frac{B}{Y}$} & 0.4 & \multicolumn{2}{c|}{} \\ \hline
\end{tabular}
\end{center}
\end{table}






\hypertarget{Results}{}
\section{Results}


\hypertarget{Conclusion}{}
\section{Conclusion}







\providecommand{\figName}{Convergence-of-the-Consumption-Rules}
\providecommand{\figFile}{cFuncsConverge}
\hypertarget{\figFile}{}
\hypertarget{\figName}{}
\begin{figure}[tbp]
\centerline{\includegraphics[width=6in]{\FigDir/\figFile}}
\caption{Convergence of the Consumption Rules}
\label{fig:\figFile}
\end{figure}














\clearpage\vfill\eject

\appendix

\centerline{\LARGE Appendices}\vspace{0.2in}




\clearpage\vfill\eject

\normalsize


\hypertarget{Computational Details}{}
\section{Computational Details}

\hypertarget{Household Bellman Equation }{}
\subsection{Household Bellman Equation}

Household i's dynamic program is

$$ V(\pmb{\mathrm{m}}_{it},\pmb{\mathrm{p}}_{it})=\max_{\{ \pmb{\mathrm{c}}_{it}\}} { U( \pmb{\mathrm{c}}_{it}, n_{it}) + \beta_{i} \not D \mathrm{E}_{t}[V( \pmb{\mathrm{m}}_{it+1} , \pmb{\mathrm{p}}_{it+1})]}$$

subject to 

\begin{align*}
 \pmb{\mathrm{m}}_{i t} & = \pmb{\mathrm{z}}_{i t}  + (1+\mathit{r}^{a}_{t})\pmb{\mathrm{a}}_{i t-1} \\
 \pmb{\mathrm{c}}_{i t}  + \pmb{\mathrm{a}}_{i t} &= \pmb{\mathrm{z}}_{i t}  + (1+\mathit{r}^{a}_{t}) \pmb{\mathrm{a}}_{i t-1}   \\
\pmb{\mathrm{a}}_{it} &\geq 0 
\end{align*} \\ \\

This can be normalized to \\


$$ V(m_{it}) = \max_{\{c_{it}\}} { U(c_{it}, n_{it}) + \beta_{i}\not D \mathrm{E}_{t}[\psi_{it+1}^{1-\rho} V(m_{it+1})]}$$

 subject to 
 
 \begin{align*}
m_{i t} &=  \xi_{it}  + (1+r^{a}_{t}) \frac{a_{i t-1}}{\psi_{it-1}} \\
 c_{i t}  + a_{i t} &= \xi_{it}  + (1+r^{a}_{t}) \frac{a_{i t-1}}{\psi_{it-1}} \\
 a_{it} &\geq 0 
 \end{align*}
 
 Here non boldface variables are normalized by permanent income $\mathit{p_{it}}$. 

e.g. $x_{it} = \frac{\mathbf{x_{it}}}{\pmb{\mathrm{p}}_{it}}$




\hypertarget{Model as System}{}
\subsection{Model as System}

$$
H_{t}(\mathbf{U},\mathbf{Z})= \begin{pmatrix} 
 Y_{t} - Z_{t}N_{t} \\ \\ 
B_{t-1} - q^{b}_{t}B_{t} + u\mho + G_{t} - \tau w_{t} N_{t} \\ \\  
i_{t} - r^{*} - \phi \pi_{t} -\phi_{y}(Y_{t}-Y_{ss}) - v_{t} \\ \\
\pi_{t} -\frac{\pi_{t+1}}{1+r^{*}} + \lambda(\mu_{t}^{p} -\mu_{p})  \\ \\
 \pi_{t}^{w} -\not D \pi_{t+1}^{w} -(\frac{1-\lambda_{w}}{\lambda_{w}}) (1- \not D \lambda_{w}) (\mu^{w} -\mu_{t}^{w}) \\ \\
    1+r_{t} - \frac{1 + i_{t}}{1+ \pi^{p}_{t+1}}\\ \\
 1+r_{t+1}^{a} - \frac{q_{t+1}^{s} +D_{t+1}}{q_{t}^{s}} \\ \\
 r_{t} - r_{t+1}^{a} \\ \\
 \frac{w_{t}}{w_{t-1}} - \frac{\Pi_{t}^{w}}{\Pi_{t}^{p}} \\ \\
 \mathcal{C}_{t}(\{r_{s}^{a} ,w_{s}, N_{s}\}_{s=0}^{s=T}) - Y_{t} + G_{t}  \\ \\
  \mathcal{A}(\{r_{s}^{a} ,w_{s}, N_{s}\}_{s=0}^{s=T}) - q_{t}^{s} - \frac{B_{t}}{1+r_{t}}  \\ \\
 \end{pmatrix} = \begin{pmatrix} 0 \\ 0 \\. \\. \\. \\ 0\\ \end{pmatrix} , \quad t=0,1 ,2,3,....
$$ \\ \\
 

 
 where \\
 
 $\mathcal{C}_{t}(\{r_{s}^{a} ,w_{s}, N_{s}\}_{s=0}^{s=T}) = \int_{0}^{1} \pLevBF_{it} c_{it}(m_{it})\, di $ \\
 
 $\mathcal{A}_{t}(\{r_{s}^{a} ,w_{s}, N_{s}\}_{s=0}^{s=T}) = \int_{0}^{1} \pLevBF_{it} \left(m_{it} - c_{it}(m_{it})\right)\, di $ \\
 
$c_{it}(m_{it})$ is the steady state normalized consumption policy for household $i$ in period t. \\ \\
 

 
 
 $\mathbf{U} = \left(Y_{t} , N_{t} ,  D_{t
 }, B_{t}, w_{t} , \pi_{t}^{p} ,\pi_{t}^{w}, r_{t} , r_{t+1}^{a}, i_{t} , q_{t}^{s},  q_{t}^{s} \right)_{t=0}^{t=T}$ \\ 

 
 $\mathbf{Z} = \left(Z_{t} ,v_{t}\right)_{t=0}^{t=T}$ \\ \\
 
 
\hypertarget{Reduced System}{}
\subsubsection{Reduced System}
 
 System Can be reduced to the following: \\ \\
 
Exogenous Variables are $ Z_{t}, v_{t}$ \\ 

Endogenous Variables are $ r_{t} , w_{t} ,N_{t}$ \\ \\

\begin{eqnarray} 
H_{t}(\mathbf{U},\mathbf{Z})= \begin{pmatrix} 
\mathcal{H}_{t,1} \\ \\ 
\mathcal{H}_{t,2} \\ \\
\mathcal{H}_{t,3} \\ \\
 \end{pmatrix} = \begin{pmatrix} 0 \\ 0 \\ 0 \\ \end{pmatrix} , \quad  t = 0, 1, 2, ..., T 
 \end{eqnarray}
 
 where \\ 
 
 
 $\mathcal{H}_{t,1}  =\mathcal{C}_{t}\left( \left \{r^{a}_{s} , w_{s} , N_{s}  \right \}_{s=0}^{s=T} \right) - Z_{t} N_{t} + G\\ \\ $

$ \mathcal{H}_{t,2}  =log(w_{t}) - log(w_{t-1}) + \left( \frac{1 - \lambda_{w}}{\lambda_{w}}(1 - \not D \lambda_{w}) \sum_{k=0}^{\infty} \not D^{k} ( \mu_{t+k}^{w} - \mu^{w}) \right) - \left(  \lambda \sum_{k=0}^{\infty} \frac{1}{(1+r^{*})^{k}} ( \mu_{t+k}^{p} - \mu^{p})\right)\\ \\ $

$ \mathcal{H}_{t,3}  =  (1+r_{t}) \left(1+ -\lambda \sum_{k=1}^{\infty} \frac{1}{(1+r^{*})^{k}} ( \mu_{t+k}^{p} - \mu^{p}) \right) \\ \\
 - \left(1+r^{*}+ \phi \left(- \lambda \sum_{k=0}^{\infty} \frac{1}{(1+r^{*})^{k}} ( \mu_{t+k}^{p} - \mu^{p}) \right) +\phi_{y} \left(Z_{t} N_{t} - Y_{ss} \right) + v_{t}\right) \\ \\ $
 
$\mathbf{U} = (r_{0} , r_{1} , ...r_{T}, w_{0}, w_{1}, ..., w_{T}, N_{0}, N_{1},...,N_{T})$ \\ 

$ \mathbf{Z} = ( Z_{0}, Z_{1},... Z_{T}, v_{0},...,v_{T}) \\ \\ $ 

$ \mu_{t}^{p} = log(\frac{1}{w_{t}}) + log(Z_{t})$ \\ 

$\mu_{t}^{w} = log(w_{t}) + log(1 - \tau_{t}) - mrs_{t} \\ \\ $


$mrs_{t} = log \left(- \frac{\int_{0}^{1}   U_{n} \left(\cLevBF_{i t}, n_{i t} \right) \ d i  }{\int_{0}^{1} \pLevBF_{it} \theta_{it} U_{c} \left(\cLevBF_{i t}, n_{i t} \right) \  di } \right) = log \left(\frac{\int_{0}^{1} \varphi \pLevBF_{it} n_{it}^{v} \ d i  }{\int_{0}^{1} \pLevBF_{it}  \theta_{it} \cLevBF_{it}^{-\rho} \  di } \right) = log \left(\frac{\int_{0}^{1} \varphi \pLevBF_{it} \left(\frac{N_{t}}{1 - \mho} \right)^{v} \, d i  }{\int_{0}^{1} \pLevBF_{it}  \theta_{it} \cLevBF_{it}^{-\rho} \  di } \right)$ \\ 

$ = log \left( \varphi \left(\frac{N_{t}}{1 - \mho}\right) ^{v}\right) + log \left(\int_{0}^{1} \pLevBF_{it}  \,  di  \right) - log \left(\int_{0}^{1} \pLevBF_{it}  \theta_{it} \cLevBF_{it}^{-\rho} \  di  \right) $


\hypertarget{Jacobian of System}{}
\subsection{Jacobian of System} 

By applying implicit function theorem to equation 10 : \\ \\

$$d\mathbf{U} =  -{\mathbf{H}_{\mathbf{U}}}^{-1} \mathbf{H}_{\mathbf{Z}} d \mathbf{Z}$$ \\ 


 $$  \mathbf{H}_{\mathbf{U}}= \begin{pmatrix} 
H_{\mathbf{u}, 0} \\ \\ 
H_{\mathbf{u}, 1}  \\ \\
. \\ \\
. \\ \\
. \\ \\ 
H_{\mathbf{u}, T} \\ \\
 \end{pmatrix} \quad \quad \mathbf{H}_{\mathbf{Z}}= \begin{pmatrix} 
H_{\mathbf{Z}, 0} \\ \\ 
H_{\mathbf{Z}, 1}  \\ \\
. \\ \\
. \\ \\
. \\ \\ 
H_{\mathbf{Z}, T} \\ \\
 \end{pmatrix}$$ \\ \\
 
 
 
$$ H_{\mathbf{U}, t}= \begin{pmatrix} 
\frac{ \partial \mathcal{H}_{t,1}}{\partial r_{0}}  & ... & \frac{ \partial \mathcal{H}_{t,1}}{\partial r_{T}} & \frac{ \partial \mathcal{H}_{t,1}}{\partial w_{0}} & ... & \frac{ \partial \mathcal{H}_{t,1}}{\partial w_{T}} & \frac{ \partial \mathcal{H}_{t,1}}{\partial N_{0}} & ... &\frac{ \partial \mathcal{H}_{t,1}}{\partial N_{T}} \\ \\ 
\frac{ \partial \mathcal{H}_{t,2}}{\partial r_{0}}  & ... & \frac{ \partial \mathcal{H}_{t,2}}{\partial r_{T}} & \frac{ \partial \mathcal{H}_{t,2}}{\partial w_{0}} & ... & \frac{ \partial \mathcal{H}_{t,2}}{\partial w_{T}} & \frac{ \partial \mathcal{H}_{t,2}}{\partial N_{0}} & ... &\frac{ \partial \mathcal{H}_{t,2}}{\partial N_{T}}  \\ \\
\frac{ \partial \mathcal{H}_{t,3}}{\partial r_{0}}  & ... & \frac{ \partial \mathcal{H}_{t,3}}{\partial r_{T}} & \frac{ \partial \mathcal{H}_{t,3}}{\partial w_{0}} & ... & \frac{ \partial \mathcal{H}_{t,3}}{\partial w_{T}} & \frac{ \partial \mathcal{H}_{t,3}}{\partial N_{0}} & ... &\frac{ \partial \mathcal{H}_{t,3}}{\partial N_{T}} \\ \\
 \end{pmatrix} $$ \\
 
  $$ H_{\mathbf{Z}, t}= \begin{pmatrix} 
\frac{ \partial \mathcal{H}_{t,1}}{\partial Z_{0}}  & ... & \frac{ \partial \mathcal{H}_{t,1}}{\partial Z_{T}} & \frac{ \partial \mathcal{H}_{t,1}}{\partial v_{0}} & ... & \frac{ \partial \mathcal{H}_{t,1}}{\partial v_{T}} \\ \\ 
\frac{ \partial \mathcal{H}_{t,2}}{\partial Z_{0}}  & ... & \frac{ \partial \mathcal{H}_{t,2}}{\partial Z_{T}} & \frac{ \partial \mathcal{H}_{t,2}}{\partial v_{0}} & ... & \frac{ \partial \mathcal{H}_{t,2}}{\partial v_{T}} \\ \\
\frac{ \partial \mathcal{H}_{t,3}}{\partial Z_{0}}  & ... & \frac{ \partial \mathcal{H}_{t,3}}{\partial Z_{T}} & \frac{ \partial \mathcal{H}_{t,3}}{\partial v_{0}} & ... & \frac{ \partial \mathcal{H}_{t,3}}{\partial v_{T}}  \\ \\
 \end{pmatrix} $$
 
 

\bibliography{\econtexRoot/BufferStockTheory,economics}

\end{document}
